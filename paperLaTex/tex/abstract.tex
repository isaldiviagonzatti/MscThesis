\chapter*{Abstract}


\noindent The management of pineapple crop residues generates large environmental and economic costs in Costa Rica. The use of these residues to produce value-added products can be beneficial for the pineapple industry and the circular bioeconomy. Although several valorisation options have been studied, none of them has been implemented at a large scale. This study presents the state-of-the-art in the extraction and valorisation of Pineapple Leaves (PAL) in Costa Rica. Using the Fuzzy Cognitive Map method, we analyse the barriers preventing the adoption of valorisation processes and the transition to a circular bioeconomy. To model a potential logistics solution to the valorisation of PAL, we present a Facility Location Problem that optimises the number and location of valorisation facilities that minimise operational costs. 

\noindent The study shows that unsustainable customs, lack of collaboration, and insufficient funding are the main barriers to circular-oriented innovation in the industry. Moreover, operational and technological barriers, especially related to the extraction of PAL from the field, hinder progress toward large-scale solutions. Government agencies are potential drivers of change, and there is a need for transparency and knowledge sharing. Awareness of the benefits of valorising must be raised to motivate investors. A decentralised valorisation operation is more suitable for biogas production considering
the spatial distribution of pineapple fields in Costa Rica and the processing capacity of biogas plants. The model presented can be used to analyse the most cost-effective operational solution for different types of valorisation techniques, including cascaded solutions. 


\vspace{1cm}

\noindent \textit{\textbf{Key words}}

\noindent Circular Bioeconomy, Crop Residue, Waste Valorisation, Pineapple Stubble, Knowledge Elicitation, Facility Location Problem
