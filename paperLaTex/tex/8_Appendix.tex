\chapter{Appendix}

\renewcommand{\thesection}{A\arabic{section}}
\renewcommand\thefigure{A\arabic{section}.\arabic{figure}}  
\renewcommand\thetable{A\arabic{section}.\arabic{table}}  

\section{FCM results}

\subsection{Experts and Stakeholders Interview Outline}
\label{interviewOutline}

\begin{itemize}

\item Name, organisation, purpose of the organisation.
\item What do you understand by Circular Economy?
\item What do you understand by valorisation of stubble?
\item What is the current situation of the valorisation of stubble in Costa Rica?
\item What factors influence the valorisation of Costa Rican pineapple stubble? Please indicate at least four. \textit{(Sector structure)}
\item What factors are influenced by the valorisation of Costa Rican pineapple stubble? \textit{(Outcome)}
\item Is there a relationship between the factors described? How would you describe those relationships? Positive, neutral, or negative?
\item Identify three key drivers that can boost the valorisation of pineapple stubble and, consequently, the circularity of the Costa Rican pineapple sector. Think of the national, international scale, factors external to the production chain.
\item Do you perceive any trends in the factors previously mentioned in the last 5 years?
\item Who are the most important actors in the stubble valorisation process?
\item Which valorisation options seem most feasible to you, and why? Think about the technological, economic, and commercial aspects of the valorisation.

\end{itemize}



\subsection{Concepts Description}

\begin{xltabular}{\textwidth}{XX}
\caption{Concepts present in the FCM and their description}
\label{conceptsList} \\

 \hline \hline  \multicolumn{1}{|c|}{\textbf{Concept}} & \multicolumn{1}{c|}{\textbf{Description}} \\ \hline 
\endfirsthead

\multicolumn{2}{c}%
{\tablename\ \thetable{} -- continued from previous page} \\
 \hline \hline  \multicolumn{1}{|c|}{\textbf{Concept}} & \multicolumn{1}{c|}{\textbf{Description}} \\ \hline 
\endhead

\hline \multicolumn{2}{|r|}{{Continued on next page}} \\ \hline
\endfoot

\hline
\endlastfoot


Regulation of stubble management &
  Regulate how the stubble can be managed (what agrochemicals can be used,  when is fire allowed, etc.). \\ \hline
Good image of the industry &
  How consumers, investors, the government, and the population in general perceive the industry. \\ \hline
Collaboration/Communication &
  Between pineapple companies, academia, government, social communities, and other industries. \\ \hline
Pollution (soil, air and water) &
  Refers to the presence of substances or particles in amounts that can be harmful to human health and ecosystems. \\ \hline
Cost of fossil fuel-based materials &
  Cost of materials used in the industry that come directly or indirectly from fossil fuels (plastics, agrochemicals, and other materials). \\ \hline
Customs of the industry &
  Practices that have been present for many years and inherited by new generations of pineapple producers. \\ \hline
Demand of PAL products &
  Demand for products derived, completely or partially, from PAL (e.g., biobased materials to replace plastics, bioenergy and biofuels, textiles). \\ \hline
Uneven terrain &
  In the pineapple plantations. In a broader sense, it refers to the inaccessibility of the terrain. \\ \hline
Land availability &
  Availability of the land to plant. As long as there is stubble in the field, the land is not available. \\ \hline
Employment &
  Number of people employed in the PA industry or in a PAL-related industry. \\ \hline
Extraction from the field &
  Extracting the PAL from the field. \\ \hline
Funding &
  Public or private funding, either national or international. \\ \hline
International instability &
  Lack of stability or predictability in the international system. It can be political instability, economic uncertainty, or military conflicts. \\ \hline
Innovation &
  Innovation related to the valorisation of PAL. Creation of new ideas, products, or methods. \\ \hline
Research &
  Carried out by universities and other scientific centres. \\ \hline
Rain &
  Precipitation in the terrain and the road network inside and outside the field. \\ \hline
Stable fly &
  Number of stable flies. \\ \hline
Government presence &
  How involved are local and national governments in the pineapple sector and in PAL-valorisation development. \\ \hline
Green consumers &
  Consumers who demand products that have undergone an eco-friendly production process and that safeguards the planets' resources. \\ \hline
Pineapple production productivity &
  Amount or weight of fruit produced per unit of land, labour, or other resources used. \\ \hline
Labour productivity &
  The efficiency with which labour is used in the production of goods. Total value of production / Total number of hours worked. \\ \hline
Import regulations and standards &
  That other countries impose to exporters from CR. \\ \hline
Profitability of pineapple companies &
  Company's ability to generate profit. \\ \hline
Business risk &
  All events that may affect or cause losses to a company within the framework of its economic activity. \\ \hline
Community's health/wellbeing &
  Health and social and environmental well-being of the communities directly or indirectly affected by PA production. \\ \hline
Industry sustainability &
  Practice of using natural resources in a way that preserves them for future generations. \\ \hline
Company  size &
  Refers to the amount of resources that the company has: hectares, workers, machinery. \\ \hline
Industry's transparency/openness &
  In relation to operations, cultivation methods, use of supplies and stubble management. \\ \hline
Use of agrochemicals &
  Chemicals used in to enhance crop growth, protect against pests and diseases, and manage the stubble (fertilizers, herbicides, insecticides). \\ \hline
valorisation of PAL &
  Activities that convert PAL into new products or materials. \\ \hline
Soil fertility &
  Ability of the soil to support plant growth. \\ \hline
Ranchers' productivity &
  It can be measured in terms of the number of animals raised, the amount of meat or milk produced, the value of their sales, or the profits they generate.

\end{xltabular}

\clearpage

\subsection{Entropy results}
\label{entropyResults}

\begin{xltabular}{\textwidth}{XXc}
\caption{Entropy values of FCM connections}
\label{entropyFCM} \\

\hhline{===}\multicolumn{1}{|c|}{\textbf{From concept}} & \multicolumn{1}{c|}{\textbf{To concept}} & \multicolumn{1}{c|}{\textbf{Entropy}}\\ \hline 
\endfirsthead

\multicolumn{3}{c}%
{\tablename\ \thetable{} -- continued from previous page} \\
\hhline{===}  \multicolumn{1}{|c|}{\textbf{From concept}} & \multicolumn{1}{c|}{\textbf{To concept}} & \multicolumn{1}{c|}{\textbf{Entropy}}\\ \hline 
\endhead

\hline \multicolumn{3}{|r|}{{Continued on next page}} \\ \hline
\endfoot

\hhline{===}
\endlastfoot

IndustryTransparency & collabComms & 1.750000 \\
academia & innovation & 1.905639 \\
agrochemicalsUse & pollution & 1.405639 \\
businessRisk & innovation & 2.155639 \\
collabComms & innovation & 0.954434 \\
communityHealth & industryImage & 1.298795 \\
\multirow[c]{2}{*}{companySize} & funding & 1.905639 \\
 & pineappleProdProfitability & 1.905639 \\
\multirow[c]{2}{*}{costFFmaterials} & palProductsDemand & 2.155639 \\
 & pineappleProdProfitability & 1.905639 \\
\multirow[c]{2}{*}{employment} & communityHealth & 1.405639 \\
 & funding & 2.155639 \\
\multirow[c]{3}{*}{fieldExtraction} & agrochemicalsUse & 1.500000 \\
 & innovation & 2.155639 \\
 & palvalorisation & 2.155639 \\
\multirow[c]{2}{*}{funding} & academia & 1.405639 \\
 & innovation & 1.811278 \\
\multirow[c]{2}{*}{govtPresence} & collabComms & 1.750000 \\
 & stubbleMgmtRegulation & 1.750000 \\
\multirow[c]{2}{*}{greenConsumers} & agrochemicalsUse & 2.155639 \\
 & palProductsDemand & 1.905639 \\
importRegulations & agrochemicalsUse & 1.561278 \\
industryCustoms & palvalorisation & 2.405639 \\
industryImage & funding & 2.155639 \\
industrySustainability & palProductsDemand & 1.905639 \\
\multirow[c]{3}{*}{innovation} & fieldExtraction & 1.298795 \\
 & laborProductivity & 1.905639 \\
 & palvalorisation & 1.561278 \\
intInstability & costFFmaterials & 1.561278 \\
laborProductivity & fieldExtraction & 2.500000 \\
landAvailable & pineappleProdProductivity & 2.155639 \\ \\
\multirow[c]{2}{*}{palProductsDemand} & innovation & 1.561278 \\
 & palvalorisation & 1.500000 \\
\multirow[c]{7}{*}{palvalorisation} & agrochemicalsUse & 2.155639 \\
 & fieldExtraction & 1.405639 \\
 & industrySustainability & 1.500000 \\
 & landAvailable & 1.561278 \\
 & pineappleProdProfitability & 2.000000 \\
 & soilFertil & 1.561278 \\
 & stableFly & 2.155639 \\
pineappleProdProductivity & pineappleProdProfitability & 1.405639 \\
pineappleProdProfitability & industrySustainability & 1.750000 \\
pollution & communityHealth & 1.061278 \\
rain & fieldExtraction & 2.155639 \\
\multirow[c]{2}{*}{stableFly} & communityHealth & 1.405639 \\
 & ranchersProductivity & 1.750000 \\
\multirow[c]{5}{*}{stubbleMgmtRegulation} & agrochemicalsUse & 2.500000 \\
 & industrySustainability & 1.905639 \\
 & innovation & 1.500000 \\
 & palvalorisation & 1.750000 \\
 & stableFly & 1.750000 \\
unevenTerrain & fieldExtraction & 2.405639 \\
\end{xltabular}



\subsection{FCM matrix}

\begin{figure}[H]
\caption{FCM weight matrix} 
\label{FCMMatrix}
\centering
\makebox[\textwidth][c]{\includesvg[width=\textwidth]{fig/matrixFCM.svg}}%
\end{figure}

\section{Supplementary material}
\label{suplmaterial}

The scripts and data used for the Fuzzy Cognitive Mapping and the Facility Location Problem are publicly available in this \underline{\href{https://github.com/isaldiviagonzatti/MscThesis}{GitHub repository}}. For more information, read the README file in the repository. 
