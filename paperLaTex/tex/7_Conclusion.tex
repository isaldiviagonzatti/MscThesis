\chapter{Conclusions and Reflection}
\label{concludeGen}

\section{Conclusions}

In \cref{neweconomy}, we described the state-of-the-art technology for extracting and valorising pineapple leaves (PAL) and the potential demand for PAL-based products in Costa Rica. Ventures to extract PAL with a combination of machinery and manual labour, although cost-effective, are scarce. Several valorisation options have been considered and studied in Costa Rica, but to this day the only active PAL-based business in the country is that of silage. Each valorisation option has specific production processes, potential demand, and applicable legislation, and each should be considered when planning a PAL-based business model. Moreover, a solution to fully recycle and valorise PAL will likely consist of a combination of valorisation options. 

Using a fuzzy cognitive map, in \cref{chapter_3lab} we analysed the stakeholders' perception of PAL valorisation. We presented the barriers preventing PAL valorisation and categorised them using a common criterion in studies of circular bioeconomy. Moreover, we discussed what drivers of change can help overcome these barriers and whose action is required. Finally, we identified specific milestones that stakeholders should achieve to transition towards a circular bioeconomy in the industry.

\Cref{FLPchapt4} serves to answer the research questions on the suitable locations and optimal spatial distribution of PAL processing plants in Costa Rica. The study of the Facility Location Problem (FLP) shows that a decentralised solution is more suitable for biogas production considering the spatial distribution of pineapple fields in Costa Rica and the processing capacity of biogas plants. Although the model results should be interpreted with care because of the limited available data, the approach can be used to analyse the most cost-effective operational solution for different types of valorisation techniques. In this sense, the FLP can help reduce uncertainty about costs and help stakeholders visualise possible solutions. Finally, the importance of accounting for environmental impacts is also recognised, and further research is recommended in this respect. 

Taking into account the conclusions of the study, our first recommendation for stakeholders and future researchers is to focus on improving collaboration and communication to find solutions faster and more efficiently and to attract funding more easily. Second, market research and awareness campaigns on the financial benefits of PAL valorisation can attract more investors. Partnerships with development aid agencies and environmental organisations have proved useful and should continue and expand. Moreover, knowledge from researchers, engineering companies, and similar industries can contribute to finding technological solutions. Finally, further studies to expand the logistical analysis of the valorisation process can be beneficial for implementing cost-competitive solutions.

We also emphasise the importance of collecting and sharing data in the early stages of any valorisation process. These data can then be used to conduct comparative studies, such as multiple-criteria decision analysis or life-cycle assessment, which can help determine the economic feasibility of different valorisation options. By collecting, sharing, and analysing data, stakeholders, accompanied by researchers, can make informed decisions and improve the valorisation process. In this sense, we highlight the importance of open data and, more broadly, open science to facilitate innovation by promoting collaboration, data access, and transparency. By leveraging the collective knowledge and expertise of researchers and innovators related to the pineapple industry in Costa Rica, we can tackle the complex challenges of PAL valorisation.

\section{Reflection}

\paragraph{A comment on qualitative analysis in economics} \mbox{}\\
Economists have traditionally used mathematical and statistical models to understand and explain phenomena and have neglected qualitative research which is perceived as less rigorous or less scientific. It is usually considered that qualitative methods are more subjective and rely more heavily on interpretation than quantitative methods. Quantitative studies are less likely to be criticized because people believe that randomly sampled data is representative of a population. This assumption is not always true, and there are limitations to generalizing beyond the population from which the sample was taken. Qualitative scholars do not usually draw random samples and are often interested in unique cases \citep{rubin2021rocking}. There are many situations in which qualitative methods can be useful for economists, and the present study is a good example of it. Gathering information on the experiences and perspectives of individuals and social systems was imperative to conduct this study and identify new research questions. Exploring the complex social phenomena that affect the valorisation of PAL in Costa Rica could not have been easily captured by quantitative data, and talking to real people was useful in discerning causality. Qualitative research can challenge economic assumptions, and it requires creativity, but it can provide flexibility when needed. In this sense, these comments are an open invitation for economists to reflect on the value of qualitative data collection and analysis. Usually, a mixed-methods approach like the Fuzzy Cognitive Map is desirable for gaining a more complete understanding of economic phenomena.

\paragraph{A comment on the political economy concerning the pineapple industry}\mbox{}\\
It is important to recognise that the environmental problems generated by the pineapple stubble in Costa Rica would not be present if pineapples were not cultivated as a monoculture under a \textit{pineapple republic} scheme. Costa Rica, like other periphery countries, was historically focused on agriculture, driven by the demand of core nations. The cultivation methods employed to meet the ever-increasing demand for tropical produce cause displacement of populations, human rights violations, deforestation, and environmental problems. Apart from the difficulties to develop economically under commodity dependence, developing countries now face the challenge of mitigating climate change with fewer resources than their developed counterparts. The structure of the global economy that created international economic inequality now also creates environmental and climate inequality. In the case of the pineapple industry in Costa Rica, the problems created by the global system affect only locals. The development of PAL valorisation as a solution to solve the environmental problems currently present is in progress, and the efforts taken by Costa Rica and its people are worth praising. 

