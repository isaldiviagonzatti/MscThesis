\chapter{Introduction}

%\begin{chapquote}{Author (\citeyear{})}
%``''
%\end{chapquote}

%\begin{displayquote}
%\textit{[T]he humid evening air is filled far and wide with the fragrance of the ripe ananas. The stalks of the pineapples, swelling with rich juice, rise between the lowly herbs of the meadow, and the golden fruit is seen shining at a distance from under its leafy crown of bluish-green.}

%\hspace*{\fill} \textbf{---}  Alexander von Humboldt
%\end{displayquote}



\section{Problem Statement}

Upon his return from his second voyage to America, Columbus presented King Ferdinand and Queen Isabella with gifts including gold nuggets, native artefacts, and various exotic birds, trees, animals, and plants, including a pineapple. Most of the pineapples rotted during the long voyage, but the king and queen tasted the unspoiled one and declared that they preferred it over all the other fruits \citep{o2013pineapple}. Centuries later, pineapples have become a common import in temperate climates. 

In 2018, the pineapple was ranked ninth as the most harvested fruit in the world, with a global production of $28 \times 10^6$ tonnes; the most harvested fruit in the world, banana, amounted to $114 \times 10^6$ tonnes \citep{fruit2020international}. The five leading producers of pineapple are Costa Rica, the Philippines, Brazil, Indonesia, and China. From these, Costa Rica is the largest exporter ($2.2 \times 10^6$ tonnes in 2021), and the second-largest producer, amounting to $2.9 \times 10^6$ tonnes in 2021 with an average increase of 6.7\% in the last 20 years \citep{FAOSTAT2022}. 

What once was one of the many riches sent from the colonies back to the mother countries is now a relevant earner for exporting countries. The pineapple industry contributes approximately 1.7\% of Costa Rica's GDP, providing 32,000 direct jobs and 120,000 indirect jobs nationally \citep{chen2020production, employmentCR}. From the permanent croplands in the country, 11\% is dedicated to pineapple, which tops the list next to coffee, palm oil, sugar cane, and banana \citep{cenagro2014}. 

To grow the \textit{fruit of kings} in the Europe of the 17th century, hothouses were employed to maintain heat by ingenious means at enormous cost \citep{o2013pineapple}. Today, different technologies coupled with international trade provide easy and inexpensive access to the \textit{queens of fruits} and its production is highly profitable for the producing countries. However, regardless of time and place, conventional farming interferes with the natural ecosystem. Playing an important role in the agriculture and economy of Costa Rica, the pineapple sector has gained domestic and international attention due to its environmental impacts that affect the nation at large.

Grown as a monoculture, pineapple farming in Costa Rica presents several environmental problems, such as building up disease pressure, reducing particular nutrients in the soil, erosion due to conventional ploughing, and contamination of soil and water sources \citep{rodriguez2020agricultural, salaheen2019organic}. In addition, the reduction in livestock productivity caused by the stable fly (\textit{Stomoxys calcitrans}) that multiplies in the crop residues has become a major problem for pineapple producers and ranchers alike \citep{alpizar2016analisis, elbersen2019costa}. This cross-sectoral problem is difficult to tackle with a fruit that produces large amounts of crop residues. Farmers usually plant between 65,000 and 80,000 pineapple plants per hectare, and these can grow to weigh between 2.5 and 3.0 kg, with a height and width of between 1 and 2 metres \citep{asim2015review}. After the last harvest, these residues, also known as stubble, are left in the field. 

%Pineapple stubble management is a challenging externality for a sector whose growth is associated with economic benefits on the one hand, and environmental concerns on the other \cite{poltronieri2016biotransformation}. 

Currently, to avoid the breeding of the stable fly in the stubble, farmers implement various practices, including drying with herbicides, burning, burying, and natural decomposition. These practices are costly both economically and environmentally. As \cite{hernandez2018impacto} note, the costs of stubble management per hectare with these practices can range between US\$ 1,000 and US\$ 2,500 depending on the method used. Furthermore, these practices increase greenhouse gas emissions, hamper pineapple production productivity, and affect communities near the fields \citep{cesarino2020fabrication,netz2007climate}. Monitoring and controlling crop residue practices is not an easy task for an industry that has expanded in the country without regulation \citep{rodriguez2020agricultural}.

The pressing issue of stubble management and its consequences has incentivised the exploration of innovative solutions, mainly focused on the valorisation of pineapple leaves (PAL). These solutions not only help to reduce the financial costs and environmental damages of current practices but also contribute to the circular economy (CE) transition and the bioeconomy. CE initiatives aim to move away from traditional ``make and dispose" models and towards systems that encourage material efficiency, and the objective of the bioeconomy is to use renewable biological resources sustainably to produce food, bioenergy and biobased goods. Thus, these models help to create more circular material systems that reduce energy and emissions \citep{IPCC_2022_WGIII_SPM}. 

The valorisation of pineapple leaves is not new; these are used for various purposes, including weaving, netting, and rope-making, which has a long history among natives of America, preceding the arrival of Europeans \citep{collins1949history, o2013pineapple}. Today's linear production model, which leads to vast amounts of stubble generation in the pineapple industry, makes valorisation solutions challenging. Different options to valorise the stubble have been tested in Costa Rica, e.g., producing fodder to feed livestock, or using the leaves of the plant to produce pineapple leaf fibre (PALF). Nevertheless, as of today, there is no large-scale stubble valorisation project in the country, and progress toward a systematic implementation seems stagnant. 

Despite the extensive desk research on technological solutions to produce biobased materials and bioenergy with PAL, socioeconomic studies on the implications of a Circular Bioeconomy (CB) transition are lacking. The adoption of innovative practices needed for this transition can be hindered by several bottlenecks, such as financial constraints, cultural and operational barriers, uncertainty about market demand, and technological limitations. Potential solutions have been identified, but many questions related to their implementation are unanswered. Moreover, there seems to be no consensus among stakeholders on the required steps for the valorisation process to take off. In this sense, an understanding of the actors, institutions, and the networks that connect them is needed. 

Finally, although many of the barriers preventing circular bioeconomy transitions are universal, solutions need to be tailored to the specific characteristics of Costa Rica. For example, the pineapple production clusters and the distinctive road network of the country require finding adaptive operational solutions. Today, there is no vision of how a large-scale valorisation operation would take place in the industry. By providing realistic solutions that can be applied systematically, stakeholders can gain momentum to make progress toward a sustainable solution more easily. Ultimately, by understanding what prevents the transition to valorisation and identifying actionable challenges, the pineapple industry in Costa Rica can advance in the agricultural CB revolution. 

\section{Objectives and research questions}
\label{researchQ}

Considering the knowledge gap identified within the problem statement, the present research considers the following aim and objectives. The aim is to help increase the sustainability of the pineapple industry by introducing circular bioeconomy principles. Two objectives have been defined for this aim: First, to explain why the valorisation of PAL has not taken off in the country, and to explain the complexity of the system in which the valorisation unfolds. The second objective is to help to understand how a large-scale valorisation process could be carried out operationally. 

These objectives are proposed given the current (early) stage at which the valorisation of PAL is in Costa Rica. Moreover, the absence of socioeconomic data and the abundance of uncertainty led us to consider an exploratory, qualitative, case-based study for our analysis.

To attain the proposed objectives, several research questions and sub-questions are proposed:

     
\begin{enumerate}

    \item A New Economy for An Old Problem
    \begin{itemize}
        \item What is the state-of-the-art technology for extracting PAL from the field and what are its estimated costs?
        \item What are the valorisation options being developed in CR and what is their development stage?
        \item What are the potential business models for the PAL?
        \item What is the demand for potential PAL-based products?
        \item What local regulations and characteristics should be considered when implementing valorisation options in CR?
    \end{itemize}
    
    \item Harvesting The Fruits Of Uncertainty
    \begin{itemize}
        \item What are the cultural, financial, market-related, operational, and technological barriers that prevent the valorisation of pineapple stubble in Costa Rica?
        \item What is needed to overcome these barriers? Whose action is required?
        \item What are the benefits and challenges of valorising the stubble?
    \end{itemize}
    
    \item The Pineapple Leaves Route
    \begin{itemize}
         \item What are suitable locations for PAL processing plants?
        \item What is the optimal spatial distribution of PAL processing plants? 
        \item Should PAL processing be centralised or decentralised? 
     \end{itemize}
    
\end{enumerate}

\section{Overview of the structure of the thesis}

The valorisation of PAL and its development in Costa Rica are explored from different angles. Thus, this paper has been divided into three chapters, following the same order as the research questions delineated above.

In \cref{neweconomy}, we provide a review of the pineapple stubble and its valorisation in the context of Costa Rica. We first give a brief explanation of the current management of pineapple stubble in the field. We then discuss the extraction of PAL from the field and its specifics. Then, the potential valorisation options and the demand for PAL are explained. Finally, a brief discussion of the local context and regulations applicable to the valorisation of PAL is provided.

\cref{chapter_3lab} explains the barriers preventing the valorisation of PAL in Costa Rica. We first provide the appropriate theory related to the circular (bio)economy and circular-oriented innovation. We continue by explaining the methodology employed for the qualitative analysis, Fuzzy Cognitive Mapping. Then the results of the study are shown and discussed. Finally, conclusions and recommendations are provided. 

The use of location analysis is useful to unravel the operational challenges of valorising PAL. This analysis is explained in \cref{FLPchapt4}. We first introduce the Facility Location Problem. Then, we explain the methodology used for the case study. The results are depicted, followed by a discussion. Conclusions and recommendations are provided. 

The scope and aim of this study have evolved throughout the research project. In addition, the author's perception and understanding of the issue at hand have transformed since the beginning of the research. Thus, in \cref{concludeGen}, general conclusions of the study are given and a reflection from the author on the subject of the study and the methods employed finalises the report. 

