\chapter*{Resumen}

\begin{otherlanguage}{spanish}


\noindent La gestión de los residuos del cultivo de piña genera grandes costes ambientales y económicos en Costa Rica. El uso de estos residuos para producir productos de valor agregado puede resultar beneficioso para la industria de la piña y la bioeconomía circular. Aunque se han estudiado varias opciones de valorización, ninguna de ellas se ha implementado a gran escala. Este estudio presenta el estado del arte de la extracción y valorización de las hojas de piña (PAL) en Costa Rica. Mediante el método de Mapa Cognitivo Difuso, se analizan las barreras que impiden la adopción de procesos de valorización y la transición hacia una bioeconomía circular. Para modelar una posible solución logística a la valorización de PAL, presentamos un Problema de Localización de Plantas que optimiza el número y localización de plantas de valorización que minimicen los costes operativos. 

\noindent  El estudio muestra que las las tradicionales y poco sostenibles prácticas, la falta de colaboración y la financiación insuficiente son las principales barreras a la innovación orientada a la circularidad en la industria. Además, las barreras operativas y tecnológicas, especialmente relacionadas con la extracción de PAL del campo, dificultan el avance hacia soluciones a gran escala. Los organismos gubernamentales son potenciales impulsores del cambio, y es necesaria la transparencia y el intercambio de conocimientos. Es necesario concienciar sobre los beneficios de la valorización para motivar a los potenciales inversores. Una operación de valorización descentralizada es más adecuada para la producción de biogás teniendo en cuenta
la distribución espacial de los campos de piña en Costa Rica y la capacidad de procesamiento de las plantas de biogás. El modelo presentado puede utilizarse para analizar la solución operativa más rentable para distintos tipos de técnicas de valorización, incluidas las soluciones en cascada. 


\vspace{1cm}

\noindent \textit{\textbf{Palabras clave}}

\noindent Bioeconomía circular, Residuos de cultivo, Valorización de residuos, Rastrojo de piña, Elicitación de conocimiento, Problemas de localización de plantas

\end{otherlanguage}
