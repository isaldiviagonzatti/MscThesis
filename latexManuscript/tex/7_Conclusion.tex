\chapter{Conclusion and Reflection}
\label{concludeGen}

\section{Summary of the main findings of the study}

1. Summarise the main contributions and findings of the study. I would divide this part in three:

2. Explanation of how the FCM helps us understand the current situation, and what the experts in the industry expect. 

3. What the FLP tells us about the location of processing plants and the possibilities to process PAL in CR. 

4. Explain the stage of development of the valorisation options, what seems to be the best way (probably a mix of many options chained to each other). Also sum up the difficulties to valorise and what is being done at the moment. 

\section{Implications of the results for the valorisation of pineapple stubble}

1. Discuss the implications of the research for practice and policy. 

2. The FCM is relevant to understand the challenges and to know what's needed from a policy point of view. How can stakeholders be incentivised. This will eventually speed up the valorisation systematically.

3. How the FLP, and in general spatial optimisation, is key to reduce costs and incentivise stakeholders to start valorising in mass. The results can reduce uncertainty. 

4. Explain how there's still a lot of improvement to be made in the technology of the pineapple valorisation options. There is technology, but most is not tailored to the PAL valorisation, and a lot of trial and error is still needed to have an out-of-the-box solution that can be applied at a large scale. 

\section{Recommendations for future research and practical applications}

2. First, recommendations on specific aspects of the system that need more attention. If it turns out that collaboration in the system is crucial (from the FCM), then researches can continue analysing how to improve this, and stakeholders should put effort in making bridges. 

3. Depending on the results from the spatial optimisation, it would be a good idea to delve deeper in this study, or to analyse more factors related to the logistics of the valorisation. 

4. Since the valorisation process is in its early stages, the creation of knowledge, the collection of data, and the spread of these two, is paramount to reach new stepping stones. For example, a MCDA, or a LCA will be valuable, but data is not present at the moment. Then we can make a comparative study of the economic feasibility of the options.

5. For stakeholders, I think it's relevant pointing out the importance of small steps. This is a process that takes time, and the system in which the valorisation unfolds is complex. Stakeholders should be incentivised to try ways to valorise, even if it's to a small extent. If everyone tries something and shares their results, more knowledge can be gathered for faster advancements.

\section{Reflection about the study}

1. If possible, I would like to give my reflection of the study, in terms of my thinking evolution throughout the thesis development.

2. First, a reflection on field work and qualitative analysis from the point of view of an economist and how challenging it can be. How qualitative analysis can be both useful and tricky to use. And what can be learnt from people in the ground as opposed to literature review/numbers. Systems are more complex than I thought, and it's important to keep this in mind when using models or doing quantitative analysis.

3. A comment on center-periphery structure, and on consumption in developed countries and the globalisation of food production. It is important to understand that the environmental problems of the stubble in CR exist because of the global dynamic of consumption and the historical dynamic of centre and periphery. Developing countries have the challenge to become sustainable in an era of climate crisis, when developed countries can presume of becoming green at a lower social cost.

