\chapter{Harvesting The Fruits of Uncertainty}
\label{chapter_3lab}

The first study of a valorisation process of pineapple leaves published in Costa Rica is a dissertation by \citep{quesada2003utilizacion}, who analyses the use of PAL as reinforcement of a polyester resin. Since then, numerous studies in the natural and social sciences related to PAL valorisation processes have been published in Costa Rica and elsewhere. Yet, after two decades of the creative dissertation's publication, the valorisation of PAL has not taken off in Costa Rica. 

This chapter is devoted to explaining the complexity of the system in which the valorisation of PAL in Costa Rica occurs. To advance in the implementation of valorisation, it is first relevant to understand why valorisation has not taken place and what the barriers preventing it are; only then it can be theorised what can be done to bring down those barriers and which stakeholders can lead the way in the circular economy.

\section{Theories on Circular (Bio)Economy}
\label{theoryframe}

 Due to the novelty of PAL valorisation and the complexity of the system in which it takes place, it is appropriate to define the concepts and discuss the theories that relate to the subject under study. There are several interlinked concepts we find relevant to discuss: the circular economy, the bioeconomy, the intersection of the last two, and the circular economy in the agricultural sector. Additionally, we discuss theories that serve to delineate the scope of our study.  Drawing from \cite{gottinger2020studying}, we describe how the Technological Innovation Systems (TIS) and similar frameworks can be used to identify the factors influencing transitions to a circular (bio)economy. The analytical framework designed by \cite{blomsma2022making} based on action recipes helps us understand how circular-oriented innovation (COI) processes unfold.

\paragraph{Circular Bio(Economy)} \mbox{}\\

 The definition of circular economy varies in the literature and \cite{kalmykova2018circular} present the commonalities found among them. The first commonality is the maximisation of the value of the resources in use, also called stock optimisation. Eco-efficiency is also commonly mentioned when defining CE, sometimes as a consequence of it, other times as a purpose, and in some cases as a synonym. Yet, the authors remind us that eco-efficiency can also be achieved in a linear economy and that CE should rather aim to be eco-effective. The latter not only focuses on minimising the cradle-to-grave flow of materials but also on generating cyclical, cradle-to-cradle processes. Another concept often mentioned is waste prevention, frequently presented as the main purpose of CE. Finally, the four Rs (Reduce, Reuse, Recycle and Recover), the mechanism for achieving CE, is another shared feature among the CE definitions. 

There are also differences between the definitions that relate to the tightness of the loop within a value chain, i.e., how closely the loops should be, and to the scope, which refers to the included resources: all physical resources or only certain sectors, products, materials, and substances. Because of these differences and because the shared features are not present in all definitions, \citeauthor{kalmykova2018circular} conclude that there is no established common ground for the variety of existing CE conceptualisations. Essentially, CE is a combination of several sustainability concepts, and it draws from other sustainability fields to construct its strategies. 

The concept of bioeconomy is commonly used alongside that of the circular economy. Their relationship and their differences as noted by \cite{carus2018circular} serve useful in our framework. Bioecomy involves the production of renewable biological resources and their conversion into value-added products, such as food, feed, biobased products and bioenergy. The objectives of the bioeconomy are the introduction of healthy, safe and nutritious food and animal feed; the provision of bioenergy and biofuels to replace fossil energy; the development of new, more efficient, and sustainable agricultural and marine practices, the mitigation of climate change through the substitution of petrochemicals by materials with lower GHG emissions and of fossil fuels by biofuels; and the emergence of new business opportunities, investment and employment to rural, coastal and marine areas, fostering regional development and supporting small-to-medium enterprises.

Both the bioeconomy and the circular economy aim to avoid using additional fossil fuels and to a more resource-efficient system. As \cite{carus2018circular} clarify, the circular economy and the bioeconomy are two different yet complementary approaches to promoting sustainability. The circular economy focuses on improving resource efficiency and reducing the use of fossil fuels by incorporating recycled materials into processes. Bioeconomy aims to replace fossil fuels with biomass derived from agriculture, forestry, and marine environments. The intersection between the two concepts, the circular bioeconomy, can be interpreted in many ways. \cite{tan2021circular} mention that the relationship between CE and bioeconomy is complex and explain that the circular bioeconomy is more than the intersection of both concepts, their combination results in a more sustainable framework. Perhaps more useful is to look at their limitations to understand their differences. CE focuses on economic and environmental benefits while ignoring the social dimension. Moreover, efficiency gains can be confronted with rebound effects in the form of increased production and consumption. As for the bioeconomy, this cannot bring the perceived environmental benefits only by substituting fossil-based resources with bio-based ones.

It is important to note that neither the CE nor the bioeconomy is focused on resources, i.e., they deal with the cycle of materials, but they ignore the relationship between this cycle and broader ecological processes and ecosystems services such as water, nutrient cycles, quality of the energy source, and protection of biodiversity and ecosystems. In this sense, we find relevant the study by \cite{velasco2021circular} on CE in the agricultural sector. As the authors define it, apart from the components of the CE defined above, the CE in agriculture should also guarantee the regeneration of and biodiversity in agroecosystems and the surrounding ecosystems. Additionally, they identify the main differentiating characteristics that need to be considered in a CE framework of the agricultural sector. These are the perishable nature of products, the close link with natural ecosystems, and the strong seasonality of production. 


\paragraph{Transition towards a Circular Bioeconomy} \mbox{}\\

In their literature review, \cite{gottinger2020studying} provide an analysis of the different theoretical frameworks used to study the transition towards circular bioeconomy (CB). They conclude that the Technological Innovation Systems (TIS) framework has empirically served as most useful to identify influential factors to transition. \cite{markard2008technological} define TIS as \textit{a set of networks of actors and institutions that jointly interact
in a specific technological field and contribute to the
generation, diffusion and utilisation of variants of a
new technology and/or a new product}. Actors can be individuals, companies, or governmental and non-governmental organisations. Institutions are the regulations and norms influencing the actions, decisions, and processes of actors. The networks can be learning networks that create bridges of knowledge, or they can be policy networks linking actors with the same beliefs and agenda. In TIS, barriers are called blocking mechanisms, which hinder technology diffusion and industry development. The concept of system weaknesses is also commonly used in the framework and is usually the focus of analysis when policy interventions are considered \citep{giurca2017forest}.

Most studies assessing the CB transition analyse the strengths and weaknesses of innovation systems, the impact of certain events on the transition, and the facilitators of the transition. Less often, studies analyse the stakeholders, their roles and expectation towards the transition, and the interaction among them. A group of studies also focus on the changes that occur within existing sectors and their role in the transition. Finally, policies and their effects were studied in specific countries or by comparing policies cross-nationally.

\citeauthor{gottinger2020studying} identify six categories which are commonly found in the literature of barriers to transition towards CB.  
First, \textit{Policy and Regulation} relates to barriers associated with existing or missing policies and regulation implementation problems.
The category of \textit{Technology and Materials} encompasses technical challenges associated with applying technology and creating products, as well as the availability of input materials and physical infrastructure. 
\textit{Market and Investment} conditions refer to obstacles related to market demand and creation and the mobilisation and availability of financial resources. \textit{Social Acceptance} includes barriers associated with public awareness, interest, and involvement, as well as opposition from the public. \textit{Knowledge and Networks} encompass barriers linked to generating and applying knowledge, as well as the existence and development of efficient networks. Finally, \textit{Sectoral Routines and Structures} contains barriers associated with the willingness and restrictiveness to change, such as risk-averse attitudes. From the analysed frameworks, TIS discovers a more extensive range of sub-categories.

\paragraph{Circular-Oriented Innovation (COI)} \mbox{}\\

A big challenge in achieving a circular bioeconomy is to come up with ways of maximising the use of resources and at the same time converting them into value-added products. Thus, innovation plays a crucial role in driving the transition to a circular economy, as it enables the development of new, more sustainable, eco-efficient, and hopefully eco-effective, business models and processes. In this sense, \cite{blomsma2022making} explain how the processes of circular-oriented innovation (COI) occur. They draw from different organisation science frameworks to examine what strategies COI practitioners find relevant and how they employ CE action recipes, i.e., relationships between concepts and actions that help to clarify how ambiguity is addressed to enable action. They structure their framework by asking the following questions: 1) What is the motivation to engage in COI? (Which residues are present, and where are they?), 2) How are circular strategies visualised to address the perceived issues? (Which circular strategies are applied, and where?), 3) Who should act to implement them? (Which actors?).

Surprising or not, \citeauthor{blomsma2022making} find that the motivation to engage in COI consists of a complex mix of factors. CE solutions aim to address multiple problems, often in the form of structural wastes where both a lack of closing loops and preventative strategies are present. Additionally, CE practitioners identify benefits, such as larger profit margins or uncoupling from raw materials that will be scarce in the future. Answering the second question, they mention that circular strategies require linking knowledge and stakeholders in a manner not previously employed in the linear economy. Because circular strategies in complex real-life cases are not usually employed individually and instead interact with each other, an understanding of these synergies is required. This relates to the fact that CE is an umbrella term that, as mentioned before and explained by \cite{kalmykova2018circular}, draws from different sustainability strategies. Finally, the answer of whose action is needed to implement CE solutions is focused on value network dependencies. The difficulties that COI practitioners encounter as CE strategies progress are related to dependency on other actors within the system. Additionally, \cite{blomsma2022making} remark on the importance of deciding when stakeholders should be involved in innovation processes. Early engagements can hinder certain CE solutions, and sometimes it is best to wait until a solution is better developed to seek collaborations. Finally, by identifying the CE action recipes, some of the barriers initially defined as important by the practitioners can later be discarded in favour of fewer but more central barriers. By focusing on a subset of barriers, innovators can take action more easily, which also allows them to take further steps more easily in the future. Nevertheless, this idea raises the question of when a circular solution can be considered sufficiently developed. In this sense, the authors recommend taking a sufficiently long-time horizon to understand circular phenomena in business. 

As mentioned before, value network dependencies play a relevant role in COI implementation. Thus, we find it useful to look deeper into how collaboration takes place in COI. \cite{brown2019companies} provide an insight into the motives, barriers and drivers that stimulate or hamper collaborative innovation within the context of CE. They divide the identified motives into intrinsic (realised for their own sake), and extrinsic (realied for external recognition).
Both can originate from personal and organisational levels. For example, responsibility for sustainability can have intrinsic and extrinsic motives, and such motives trigger collaboration with other actors if both parties feel alignment between their motivations. Moreover, the recognition of interdependence also stimulates collaboration. The complexity of CE strategies and the dispersion of knowledge among stakeholders drives this interdependence. 

Another motive driving collaboration is the necessity to find suitable experiment arrangements. These arrangements break down complex systems into manageable projects. Then, experimentation helps in creating knowledge, and in engaging stakeholders to develop evidence that helps overcome barriers to adopting CE strategies. Testing at scale is important to identify unintended or unexpected impacts in the system, and collaboration is necessary to share the potential risks and costs. The last motive stimulating collaboration identified by \citeauthor{brown2019companies} is the need to put the business model into action. This motive is not as developed because technical innovation is usually more advanced than market/business model innovation. Collaboration is needed to develop all the operations required for CE strategies, but there is less collaboration in this area due to competition. In this sense, a clear barrier to collaboration in the context of COI is the contradiction of companies wanting to share but also protect knowledge. Moreover, sharing economic rewards becomes more difficult as companies prioritise individual returns over the shared benefits drawn from the project. This creates a cultural barrier that can hinder the progress of collaborative innovation efforts beyond the experimental phase. 

If the culture among organisations involved in the COI is not aligned, the shared CE objectives will not develop. The challenge is to increase internal motivation and change the culture before even achieving evidence of CE. As concluded by the \citeauthor{brown2019companies}, COI is confronted with the challenge of transitioning from exploring new market opportunities and closed-loop experiments to initiating societal transformations through larger-scale collaborations. This necessitates overcoming barriers related to organiational mindsets and collaborative knowledge sharing. The latter requirement is a shared concept among CE frameworks. For example, \cite{antikainen2016framework} highlight the importance of the interaction between stakeholders when building CE business models. Their study corroborates the idea that CE business models are not isolated but rather integrated into a system of business models that together close a material loop. COI, in this sense, requires the collaboration and communication of many parties.

Inventions do not necessarily bring innovation. When we analyse the valorisation of PAL, we can think of it merely as the invention of a new product and the recycling of agricultural residues, or we can see it as part of an innovation process driven by many stakeholders for a sufficiently long period of time to transition to a more circular economy. Whether PAL valorisation remains an invention or progresses to become an innovation depends on the capacity of the industry and stakeholders to overcome the barriers preventing its implementation and to exploit its linkages to the circular bioeconomy revolution.

\begin{comment}
....................

1. Here I want to explore the topic of innovation in the Circular Economy. Business models are changing, and innovation (technological and businesswise) is required. It is important to understand what drives this innovation, and what stages an industry must go through to become circular. Why does innovation occur at a certain timescale? (why is it stagnant in some cases?). 

3. In the specific case of agricultural residues valorisation, there is a list of critical success and risk factors, which could be classified in five categories: (1) technical and logistic, (2) economic, financial and marketing, (3) organisational and spatial, (4) institutional and legal, and (5) environmental, social and cultural factors. This helps to set expectations for the study.

4. I also think that providing this framework and locating the case of CR within this framework will not only motivate the methods, but also look towards further research, including methods we could not implement because of unavailable data or lack of advancements in the development of solutions.

5. This process of innovation helps to explain the stage in which valorisation of stubble is at. It also helps to explain the objective and purpose of this thesis.

6. It could be good to make a diagram of how knowledge is created in a certain domain and what stages are needed to attain a goal. In this case, some stages need be surpassed so that we can talk about economic feasibility. An example that has gone through all stages: collect plastic from the ocean, or even renewable energy. This is a good argument to why you are doing this research in this way, at this stage.

Diffusion of innovation theory: This theory explains how and why new ideas and technologies spread through social systems over time. 

Technology acceptance model: This theory can be used to understand the factors that influence the adoption and usage of new technology. In the context of transitioning to a circular management of pineapple stubble, the technology acceptance model can be used to identify the technological and cultural barriers that prevent stakeholders from adopting and utiliing circular practices.

Resource dependence theory: This theory can be used to explain how organiations are dependent on their environment for resources and how they manage their dependencies. In the context of pineapple stubble, resource dependence theory can be used to understand the interdependencies of actors in the industry and how they may affect the adoption of circular practices.

Institutional theory: This theory can be used to explain how institutional norms and values shape the behavior of actors in an industry. In the case of pineapple stubble, institutional theory can be used to understand how cultural beliefs and practices, as well as regulatory frameworks, may inhibit the adoption of circular practices.

Socio-technical systems theory: This theory can be used to understand the interactions between social and technical components of a system and how they influence each other. In the context of pineapple stubble, socio-technical systems theory can be used to understand how social, cultural, and technical factors interact to prevent the adoption of circular practices.

Prospect theory: This theory explains how people make decisions under risk and uncertainty. In the context of pineapple stubble, prospect theory can be used to understand how risk-taking aversion among companies may inhibit the adoption of circular management practices.

Real options theory: This theory can be used to understand how firms make investment decisions under uncertainty. In the context of pineapple stubble, real options theory can be used to understand how companies may perceive the uncertainty of the benefits of adopting circular management practices as a barrier to their adoption.

Innovation adoption model: This theory can be used to understand the factors that influence the adoption of innovations by organisations. In the context of pineapple stubble, innovation adoption model can be used to identify the factors that influence the adoption of circular management practices, including the perceived risks and uncertainties associated with the transition.

\end{comment}

\section{Methodology}

\subsection{Fuzzy Cognitive Maps for knowledge elicitation and analysis}

%A qualilative approach based on participatory knowledge is adequate to evaluate the local context pertaining to PAL valorisation, and to its complex interrelations. Qualitative research allows...\cite{rubin2021rocking}

A lot of knowledge is created in the process of innovation. Many times, this knowledge can be dispersed, or fuzzy. The fuzzier the knowledge representation, the easier the knowledge acquisition, but also the harder the knowledge processing. In such cases, Fuzzy Cognitive Mapping (FCM) serves as a great tool to elicit this fuzzy knowledge because it allows fuzzy degrees of causality between causal concepts \citep{kosko1986fuzzy}. 

FCM is a technique that builds quasi-quantitative models from the knowledge of interconnected variables in a system. It was first introduced by \cite{kosko1986fuzzy}, who presented FCM as a tool for modelling complex systems and decision-making. Fuzzy Cognitive Maps are composed of a set of nodes, which represent the variables, or concepts, of the system, and a set of links between them, which represent the relationships between the variables. Each link is associated with a weight, representing the strength of the relationship between the variables. These weights can be either positive or negative and, thus, variables can ``decrease'' or ``increase''. A simple example of a FCM with three concepts and four connections is presented in \cref{example_fcm}. The connections between the variables in the system represent causal influence. The exploration of how these causal influences propagate through the system when it is subject to change or intervention is the main objective of FCM \cite{barbrook2022systems}.

\begin{figure}[H]
\caption{Example of a Fuzzy Cognitive Map}  
\label{example_fcm}
\centering
\includesvg[inkscapelatex=false,width=8cm]{fig/fuzzyExample}
\end{figure}

There are two main approaches to using FCM. The ``causal" approach implies that the strength of links between concepts represents how certain or not the experts are that a factor causes, or suppresses, another. The ``dynamical" approach models the propagation of effects of one concept on another, resulting in a representation of the relative magnitude of changes in concept values. This approach allows us to understand which concepts are most important or influenced in a system. In our study, we use the dynamical approach because it reflects a widespread usage of FCM in a participatory manner and allows for a better interpretation of cognitive maps.

The FCM modelling technique allows for more nuanced modelling of relationships that better reflect the uncertainty and ambiguity that is often present in real-world systems. As \cite{ozesmi2004ecological} explain, FCM can involve local people who typically possess a comprehensive understanding of the ecosystem and whose participation and input can be crucial for informed decision-making and gaining acceptance from the public for the proposed solutions. Furthermore, ``wicked" environmental problems that involve many stakeholders and which have no easy solutions can benefit from a model that brings together the knowledge of different experts from different disciplines and compares their perceptions. In this way, FCM can help to understand the advantages and disadvantages of possible decisions. Finally, it is important to highlight the benefits of the interaction between the researcher and the stakeholders throughout the analysis. As a participatory approach, FCM allows for recursive feedback, which helps to model an accurate representation of reality. Moreover, this approach provides the stakeholders with the opportunity to reflect on the problem at hand; the process of building a Fuzzy Cognitive Map is as important as its results. 

The use of FCM in the field of environmental studies is extensive, covering climate change \citep{kontogianni2012you, singh2014livelihood, reckien2014weather}, deforestation \citep{kok2009potential}, fire ecology \citep{devisscher2016anticipating, eriksson2022using}, pollution \citep{anezakis2016fuzzy, salberg2022assessing, ozesmi2003participatory}, renewable energy \citep{jetter2011building, alipour2019characteristics, kyriakarakos2014fuzzy}, urban ecology \citep{assunccao2020rethinking, olazabal2016use}, water use \citep{giordano2005fuzzy, kafetzis2010using}, and waste management \citep{falcone2020use, morone2021using, kokkinos2018fuzzy, konti2018exploring}. 

There are many ways of developing and applying Fuzzy Cognitive Maps, but the structure is regularly the same. We find the six steps proposed by \cite{edwards2021building} and  illustrated in \cref{stepsFCM} to be suitable for our case. The authors implement an episodic and asynchronous method with the participation of stakeholders on two occasions. The process starts by determining the scope and objective of the study, resulting in the design of an interview outline. The second step is the selection of stakeholders. Then, stakeholders are consulted to generate knowledge by means of in-depth individual interviews. The fourth step consists of the qualitative aggregation of concepts, in which the interviews' output is analysed and harmonised. In the fifth step, the stakeholders participate for a second time to weigh the connections identified in the previous step, resulting in individual fuzzy cognitive maps. The last step involves combining the responses to generate the aggregated FCM and further analyse it. In the following sections, the implantation of these steps for our case is explained in detail.

\begin{table}[H]
\centering
\resizebox{\textwidth}{!}{
\begin{threeparttable}
\caption{Fuzzy Cognitive Map building steps and products}
\label{stepsFCM}
\begin{tabular}{lll} \hline \hline 
       & Process                           & Product                                            \\ \cline{2-3} 
Step 1 & Definition of objective and scope & Interview Questions                                \\
Step 2 & Stakeholder selection             & List of Participating Stakeholders                 \\
Step 3 & Knowledge generation              & Original concepts and connections                  \\
Step 4 & Qualitative aggregation           & Generalised labels for concepts, added connections \\
Step 5 & Weighting connections             & Individual FCMs                \\
Step 6 & Quantitative aggregation          & Aggregated FCM   \\ \hline   \hline     
\end{tabular}

\begin{tablenotes}
      \small \item Adapted from \cite{edwards2021building}
\end{tablenotes}
\end{threeparttable}%
}
\end{table}



\subsection{Definition of objective and scope}

We begin by defining the objective and the scope of the FCM study which, as explained by \cite{edwards2021building}, both guide stakeholder identification and the questions posed to elicit knowledge on the system. In this chapter, we try to achieve the first objective defined in the introduction: to understand why the valorisation of PAL has not taken off in Costa Rica. Thus, the scope, which refers to the study area we try to describe, is at the national level. The main question is \textit{What are the barriers preventing the valorisation of PAL in Costa Rica?} Thus, PAL vaporisation is the first and central concept of the FCM. We finally define the interview questions to be asked to the stakeholders in the third step. It was decided to formulate open-ended questions to allow the interviewees to elaborate on their answers; this is often the domain of qualitative research \citep{barbrook2022systems}. The interview template is shown in \cref{interviewOutline}.
  
\subsection{Stakeholder selection}

When implementing FCM, it is important to represent all the people who affect or are affected by the system under study. A good way of determining whether the population to be represented has been sampled sufficiently is to check the number of new concepts added by each new participant. \cite{ozesmi2004ecological} examined accumulation curves of the total number of variables versus the number of interviews, as well as the number of new variables added per interview, using Monte Carlo techniques. Their results, depicted in \cref{accumFCM}, show that as the number of interviews increases, the number of new variables decreases and that the total number of variables increases at a decreasing rate. This behaviour is normal if we consider that most stakeholders in the system share the same vocabulary about the subject of inquiry. A similar result can be found in \cite{morone2021using}, whose research demonstrates how the generation of new variables declines rapidly. Yet, \cite{ozesmi2004ecological} note that we can expect one or two new variables to be mentioned for each new interview. 

\begin{figure}[h]
\caption{Accumulation curves of concepts vs. interviews} \label{accumFCM}
\begin{subfigure}[b]{0.45\textwidth}
  \centering
  \includegraphics[width=\textwidth]{fig/numVars.jpg}
\caption{Number of variables vs. number of maps} 
  \label{accumFCM:sub1}
\end{subfigure}%
  \hfill
\begin{subfigure}[b]{0.45\textwidth}
  \centering
  \includegraphics[width=\textwidth]{fig/newVars.jpg}
\caption{Number of new variables added per map}    
  \label{accumFCM:sub2}
\end{subfigure}
%\captionsetup{justification   = raggedright, singlelinecheck = false}
\caption*{\textit{Source:} \cite{ozesmi2004ecological}}
\end{figure}

It is useful to think of categories when selecting stakeholders in the system, and then identify individuals who fit the categories. As such, in our case study, we defined three broad categories: research, government, and industry. Once contact was made with one individual within each category, more participants were identified via snowballing. 

\subsection{Knowledge generation}

\cite{edwards2021building} highlight how important it is that the researcher designs and executes the stakeholder engagement during their participation in the process of knowledge elicitation so that their perspective is represented in the final product of the FCM. At this stage, the researcher decides the balance of co-production, i.e., how much stakeholder input and researcher input are used in the process. In-person, semi-structured interviews were held separately with each identified stakeholder. It is important to have a structured interview process that allows for the collection of detailed information, while also giving the interviewee the freedom to share the information they deem most important. At the interviews, stakeholders were first introduced to the objective and the scope of the study. Then, the predefined questions were asked and, depending on interviewees' responses, additional questions were posed to clarify or augment the information provided. No concepts were provided to the stakeholders, but it was ensured that the focus was kept on the valorisation of PAL. Interview sessions lasted between 40 and 120 minutes, were conducted between August and November 2022, and were recorded when respondents gave their consent. 

\subsection{Qualitative aggregation}

The recording of the interviews was transcribed using Whisper, a general-purpose speech recognition model. The script used for the transcription can be found in the Supplementary Material. The transcription of the interviews together with the notes taken by the researcher was used to create a list of concepts mentioned by the participants. In cases in which interviewees defined the same concept using different vocabulary, the definitions were grouped into one concept. As concepts were identified, connections were established as well. 

In \cref{process_fcm} we demonstrate how the following statement made by one of the stakeholders was converted into four concepts and three connections: ``[W]hat we are looking for with the recovery of stubble is ... converting something that today is waste into a value-added product. [A]n economic benefit, but also a social, environmental benefit. For example, if we avoid the fly problem, we already have an important environmental benefit." This is a straightforward example, as it includes the central subject of study, the valorisation of PAL, and its main consequences. 

In the figure, it can be observed how vocabulary harmonisation and concept grouping take place. In step a, identified concepts, and connections between them, are extracted from the interviewee's statement. Then, the concepts \textit{Recovery of stubble} and \textit{Value-added products} are grouped into \textit{PAL valorisation}. \textit{Economic benefit} is translated into \textit{Increase profitability of the pineapple producers}, as producers are assumed to be the ones valorising the PAL in most cases. Moreover, economic benefit is a broad concept that can have different definitions for different stakeholders. For the second round of stakeholders' participation, it is important to use concepts that are clear and that convey a similar definition to everyone. For this harmonisation process, analysis of commonly used terms used in the literature is also useful. The same explanation is valid to the translation of \textit{Environmental benefits} into \textit{Increase Community’s Health/Wellbeing}. 

In the third step, it can be observed how the concepts take their final shape. The impact previously denoted in words, increase and reduce, are now connections, represented by positive and negative signs. It is worth noting how this affects the relation between concepts: in step two, Community’s Health/Wellbeing is enhanced because stable fly is avoided as valorisation of PAL takes place. In the FCM, this is represented by two negative effects, one from valorisation of PAL to stable fly, and another from the latter to the Community’s Health/Wellbeing. This way, the effect of PAL valorisation on Community’s Health/Wellbeing is transmitted through a negative effect to and from the  (presence of) stable fly. This simple example also shows how important it is to use concepts that behave like variables, i.e., that can be thought to increase or decrease. 

The concepts and connections shown in \cref{process_fcm} are only part of the FCM built for our case, and the displayed concepts can affect or be affected by many other concepts. The aggregation and harmonisation process exemplified here needs to be repeated by revisiting statements and checking the logic and internal consistency within the concepts and connections. As more statements from different stakeholders are analysed, concepts are renamed, and connections are added or deleted. Finally, we reach a map that represents how the stakeholders perceive the system dynamics and that can be understood with ease. 

\begin{figure}[H]
\caption{Example statement processing to build FCM concepts and connections}
\label{process_fcm}
\centering
\includegraphics[width=\textwidth]{fig/processFCM.pdf}
\end{figure}


\subsection{Weighting connections}

After the concepts and connections are identified, the next step is to assign weights to the connections. For this purpose, a second round of participation from the stakeholders takes place via an online questionnaire. The online survey software Qualtrics was used to make the questionnaire form. For each connection, stakeholders were asked to assign a value to the strength they think exists between concepts. In the previous step, the sign of the connections was assigned based on what the majority of the statements from the interviews indicated. Thus, stakeholders were asked to indicate only the value of the connections. At the end of the questionnaire, respondents could comment on relationships or signs which they thought were incorrectly identified, or add new concepts and connections they thought were missing. At this point, it is relevant to remind the reader that the FCM developed in this study is the so-called dynamical FCM, which means that the values represent the propagation of effects of one concept on another and not the measure of certainty that the stakeholder has of the connection. 

All questions in the questionnaire followed the same structure ``If concept A increases, how much does concept B increases (decreases), on a scale from 1 to 5? (1 being increases (decreases) little and 5 being increases (decreases) a lot)". Initially, qualitative values --- Very High, High, Medium, Low, Very Low --- were used in the questionnaire, but it was soon realised that this, together with the sign of the connection, was confusing to respondents, and that a numerical scale, coupled with the terms increase and decrease to represent the sign of the connections, was simpler to interpret. The best way to formulate the questions is the one that works for the case and the target group, and example of both qualitative and quantitative values can be found in the literature (e.g., see \cite{morone2021using} for the former and \cite{olazabal2016use} for the latter). A glossary containing all concepts and a definition was provided with the questionnaire in case respondents had doubts about what a concept term meant. At the end of the questionnaire, the output of the previous step, the visual map with all identified concepts and connections, was also provided to aid respondents in identifying missing concepts and connections. After four weeks of sending the questionnaire, the responses were collected to proceed with the final step in the process. 

\subsection{Quantitative aggregation}

For the purpose of the quantitative aggregation, the FCMpy package, a Python package for building FCMs and implementing scenario analysis, was used \citep{mkhitaryan2022fcmpy}. The data extracted from the questionnaire responses was transformed to fit the structure required to use the package. The script produced to handle the data can be found in \cref{suplmaterial}. Then, we aggregate the responses by converting the categorical ratings to numerical weights. Sometimes, researchers use a scale to weight the consistency of stakeholders' answers, giving more weight to experts who are believed to be more knowledgeable. In our case, we have made the assumption that all individual FCMs are equally valid, and the same weight was applied to all maps. The aggregation of individual FCMs can be performed in several ways, the averaging of all individual matrices being the simplest one \citep{jetter2014fuzzy}. This aggregation approach is usually followed by normalisation of the values to narrow the connections' weights in the range [-1, 1]. Several examples using this approach can be found in the literature (see \citep{lopolito2020combined, morone2021using, morone2019promote}. In other cases, authors do not explicitly explain the method used, and it is simply mentioned that the aggregation is handled by the software selected to conduct the analysis \cite{konti2022determinants, kokkinos2020circular, falcone2020use}. In our case, we implement fuzzy logic to perform the aggregation of individual FCMs, as recommended by the developers of the FCMpy package. The aggregation via fuzzy logic, although not common, has been used in the past \citep{nasirzadeh2020modelling, amini2022combined}. This method has the advantage of using membership function maps, which are useful when it is hard to define a specific cut-off value for a linguistic term \citep{wang2015study}. Thus, this technique is well-suited to FCM applications where the input is created from human expert knowledge. 

The conversion to numerical weights requires four steps: 1) define the fuzzy membership functions, 2) apply a fuzzy implication rule, 3) combine the membership functions, and 4) defuzzify the aggregated membership functions to derive the numerical causal weights \citep{mkhitaryan2022fcmpy}. In Step 1) we define a triangular membership function that represents the linguistic terms, as can be observed in \cref{trapmf}. Step 2) requires calculating the proportion of the answers to each linguistic term for a given concept, and then applying a fuzzy implication rule to allocate the weights to the corresponding membership functions. The Mamdani minimum fuzzy implication rule is used, which basically applies a function to compute the element-wise minimum of array elements to cut the membership function at the level of endorsement, as shown in \cref{mamdani}. The aggregation of the membership function takes place in step 3), using an fMax function, simply computing the element-wise maximum of array elements, in this case to ``merge" the membership functions, resulting in a single shape representing the level of endorsement for a particular connection. Finally, we defuzzify the aggregated functions by means of the centre of gravity method, resulting in a single value for each concept, as shown in \cref{defuzz}. At the end of the aggregation of individual FCMs via fuzzy logic, we obtain a value for each connection representing its social, or aggregated, weight. In practice, the result is a matrix $n \times n$ whose element $ E_{ij} $ indicates the value of the weight $ W_{ji} $ between concept $ C_{j} $ and concept $ C_{i} $. 



\begin{figure}\centering
\subfloat[Triangular membership functions]{\label{trapmf}\includesvg[width=.45\linewidth]{fig/trapmf.svg}}\hfill
\subfloat[Fuzzy implication rules]{\label{mamdani}\includesvg[width=.45\linewidth]{fig/mamdani.svg}}\par 
\subfloat[Defuzzification of the aggregated membership functions]{\label{defuzz}\includesvg[width=.45\linewidth]{fig/defuzz.svg}}
\caption*{Adapted from \cite{mkhitaryan2022fcmpy}}
\label{FCMpy}
\end{figure}



With the aggregated matrix, we can now perform a dynamic analysis of the FCM. As \cite{edwards2021building} mention, in a mathematical sense, the output from the analysis is static rather than dynamic, so they adopt the term ‘quasi-dynamic’ to indicate the dynamic character of the interpretation of system changes. We agree with this caveat and find it relevant to mention. This quasi-dynamic analysis allows us to see where the system will go if things continue as they are, i.e., to determine the steady state of the system \citep{ozesmi2004ecological}. The steady state value taken by each concept reflects its importance within the system according to stakeholders' knowledge, and provides an idea of the evolution of the system in under current circumstances \citep{lopolito2020combined}. 

To compute the steady state of the system, a vector of initial states of variables (usually set to 0 or 1) is first multiplied with the aggregated adjacency matrix of the FCM. Then, the resulting transformed vector is repeatedly multiplied by the adjacency matrix and  transformed until the system converges to the steady state. To maintain the values in the range [0,1] and reach the steady state, an inference method, including a threshold function, is used in each iteration:

\begin{equation}
\label{equation1} 
A_i^{t+1} = f \left( A_i + \sum_{j=1}^{n} A_j^t W_{ji} \right), 
\end{equation}

where $A_i^{t+1}$ is the value of concept $C_i$ at simulation
step  $t+1$, $A_i^{t}$ is the value of concept $C_i$ at simulation
step t, $A_j^{t+1}$ is the value of concept $C_j$ at time t, $W_{ji}$ is
the weight of the interconnection from concept $C_j$ to
concept $C_i$, and $f$ is the Sigmoid bounded monotonic increasing function in the form

\begin{equation}
\label{equation2}  
f(x) = \frac{1}{1+e^{-\delta x}}, \quad  x \in \mathbb{R}, 
\end{equation}

where $x$ is the defuzzified value and $\delta$ is a steepness parameter for the Sigmoid function. Note that this non-negative transformation allows for a better understanding and representation of activation levels of variables \citep{ozesmi2004ecological}. The inference method shown in \cref{equation1} is the modified Kosko function, a modified version of the Kosko rule suitable when we require the updating of the activation value of concepts that are not influenced by other concepts \citep{sujamol2018study}. A rescaled inference rule is also included in most FCM packages and software programs, although its properties are not well explained. To the best of our knowledge, the rescaled inference rule was introduced by \cite{papageorgiou2011new} to avoid conflicts in which the initial values of concepts are 0 or 0.5, or in cases where initial values of concepts are not known. Yet, this inference method was applied in the context of health informatics, and thus we decided not to consider it for our study. 



It is important to note that iterations are not related to time. This property allows an interpretation of the dynamics of the different factors relative to the other factors, or relative to other system descriptions \citep{edwards2021building, diniz2015mapping}. In this sense, it is possible to evaluate different scenarios and outcomes by asking ``what-if" questions and simulating different conditions or policy choices. This can be used to compare what policy decisions or changes in the system would have the largest effect on the variables of interest.

\section{Results and Discussion}

        
\subsection{Descriptive analysis}
\label{descriptiveFCM}

A total of 14 experts participated in the study. Three are categorized as research-related, one as government, and 10 as industry-related. In the latter, we can find companies directly related to pineapple production and companies that are involved in PAL valorisation in some way. All stakeholders engaged in the first round of participation, which consisted of one-to-one, in-person interviews. The online questionnaire, which took place in the second round of participation, was responded to by only half of the stakeholders. Four additional pineapple producers, two government agencies, and one PAL-valorisation-related company were contacted to take part in the study, but no answer was received. A list of the stakeholders, their affiliation and role, and their engagement in the participation rounds can be found in \cref{expertsList}.

\begin{table}[ht]
\centering
\resizebox{\textwidth}{!}{
\begin{threeparttable}
\caption{Stakeholders’ profile, role, and participation}
\label{expertsList}
\begin{tabular}{lcp{0.5\textwidth}cc} \hline \hline \\

Group & \begin{tabular}[c]{@{}c@{}}Respondent \\ Code\end{tabular}  & Role &\begin{tabular}[c]{@{}c@{}}Participation \\ in 1st round\end{tabular}  & \begin{tabular}[c]{@{}c@{}}Participation \\ in 2nd round\end{tabular}        \\ \hline
    Research & R1  & Works at university conducting research in  pineapple valorisation options      & Yes & Yes \\
                           & R2  &    Works at university conducting research in  pineapple valorisation options                                                                                              &               Yes        & Yes \\
                           & R3  & Agri-food research organisation involved in the design of an extraction machine                   &   Yes                    & No  \\ Government                 & G1  & Agency of the Ministry of Agriculture in charge of protecting agricultural resources from pests. &    Yes                   & No  \\
Industry & I1  & Small-scale farmer considering valorisation options                                              &           Yes            & Yes \\
                           & I2  & Large-scale farmer                                                                               &    Yes                   & No  \\
                           & I3  & Medium-scale farmer with various PAL valorisation projects                                       &    Yes                   & No  \\
                           & I4  & Large-scale farmer with a PAL valorisation business                                              &   Yes                    & Yes \\
                           & I5  & Large-scale farmer with a R\&D team researching valorisation options                             &    Yes                   & Yes \\
                           & I6  & Association of pineapple producers                                                               &      Yes                 & No  \\
                           & I7  & Company working on field extraction machine                                                      &     Yes                  & Yes \\
                           & I8  & Company marketing PAL as fodder                                                                  &      Yes                 & No  \\
                           & I9  & Company producing and marketing PALF                                                             &     Yes                  & No  \\
                           & I10 & State-owned enterprise that developed a PAL-based biogas plant                                                               &  Yes                     & No  \\ \cline{1-5} 
Number of participants  & & &  14 & 7 \\ \hline \hline
\end{tabular}
\end{threeparttable}%
}
\end{table}


The generation of knowledge by stakeholders in the interviews resulted in 32 concepts and 52 connections. The diagram representing the connections is presented in \cref{FCMdiagram}. A list with the description of the concepts, which was also shared with the stakeholders in the questionnaire, is shown in \cref{conceptsList}. A section to add comments was provided on the online questionnaire, and valuable feedback was given by three stakeholders. Stakeholders mentioned that some concepts were too broad and that narrowing the definition can make the connections clearer. They also mentioned a disagreement with the effect of the concept \textit{Profitability} on the concept \textit{Sustainability of the industry}; indeed, some stakeholders defined this relationship as positive in the interviews, but the majority stated it was negative. Finally, the stakeholders emphasised the importance of the transparency of the industry for the collection of data needed to conduct large-scale valorisation studies, and that the results of the small-scale studies that have been developed cannot be extrapolated. No further concepts or connections were added in this comment section. 

\newpage

\begin{landscape}
\begin{figure}[H]
\caption{FCM resulting from interviews}  
\label{FCMdiagram}
\centering
\includegraphics[width=23 cm]{fig/diagram.drawio.pdf}
\end{figure}
\end{landscape}

\clearpage

Most concepts were mentioned by more than one stakeholder. The most mentioned concepts, apart from \textit{valorisation of PAL}, were \textit{Extraction from the field}, \textit{Innovation}, \textit{Use of agrochemicals}, and \textit{Government presence}. The least mentioned concepts are \textit{Ranchers' productivity}, \textit{Soil fertility}, and \textit{International instability}. It is also useful to look at the entropy, defined as 

\begin{equation}
\label{entropyEq}
E(R) = - \sum_{i=1}^{11} p_i \times log_2(p_i),   
\end{equation}

for a relationship \textit{R}, where $p_i$ is the proportion of the answers (per linguistic term) about the causal relationship between two concepts. As an example, let us take the entropy for the effect of \textit{PAL valorisation} on \textit{Sustainability of the industry}. For this relationship, one linguistic term was chosen by three respondents, another two terms were chosen by two respondents each, and the remaining eight terms were not chosen. This translates to $1 \times (- \frac{3}{7} \times log_2(\frac{3}{7})) + 2 \times (- \frac{2}{7} \times log_2(\frac{2}{7})) = 1.50 $ . The larger the entropy, the less agreement there is between experts on a particular relationship. Of the 52 connections in the FCM,  \textit{Collaboration/Communication} on \textit{Innovation} was the connection with the lowest entropy (0.954), and \textit{Labour Productivity} to \textit{Extraction from the field} and \textit{Stubble Management Regulation} to \textit{Agrochemicals Use} the two with the largest entropy (2.50). The median and the mean of the entropy values are 1.75 and 1.79 respectively. 

It is also valuable to describe the structure of the FCM by using graph theory and network analysis. Specifically, we can look at the network density, the in-degree and out-degree, and the centrality of the network. The density tells us how highly the concepts are connected to each other in the network. The in-degree and out-degree measure the total weight of relations entering and exiting a particular concept. Finally, centrality represents the sum of in- and out-degrees, determining the role of the individual variables within the system.

The network density can be calculated easily, it is simply the number of connections in the system over the potential connections. The network density of our FCM is $52/\frac{32\times31}{2}=0.104$, i.e.,  a density of 10.4\%. \cite{ozesmi2004ecological} note that a low density indicates that interviewees see a low number of causal relationships among the concepts, which translates to fewer options to change things in the system, and \cite{jetter2014fuzzy} explain that it can be interpreted as an undesirable loss of information on connections or as a desirable focus on less but truly important connections. The out and in-degree indices are presented in \cref{degreeFCM}. The most influential variables in the system are \textit{Innovation}, \textit{Extraction from Field}, and \textit{PAL valorisation}. Similarly, the most influencing variables are \textit{PAL valorisation}, \textit{Stubble Management Regulation}, and \textit{Innovation}. Since Innovation and PAL valorisation have large in- and out-degree, they are considered important in the transition process of the system.  

There are eight ``senders", i.e., variables with zero in-degree and positive out-degree, meaning that their role is to stimulate the rest of the system. The senders are represented with a circle in the \cref{FCMdiagram}. It is relevant to notice that senders are usually used as policy drivers for the intervention scenarios \citep{morone2021using}.  In our case, we consider the following concepts as policy drivers: \textit{Green Consumers}, \textit{Government Presence}, \textit{Import Regulations}, and \textit{Industry Transparency}. From these policy variables, government presence and Green Consumers have the largest out-degree value, meaning that they can have the largest impact in the system. Similarly, there are two \textit{receivers}, variables with positive in-degree and 0 out-degree, they receive input from other variables and can be used as final monitors of the system. The two receivers are \textit{Soil Fertility} and \textit{Ranchers productivity}. The rest of the variables, those with non-zero in-degree and out-degree, are “transmitters”, and they keep the system connected. Usually, receivers are outcome variables, they reflect the response of the system to interventions. In our case, outcome variables can be both receivers and transmitters, and they are \textit{Ranchers productivity, Pineapple Producers' Profitability, Community's Health/Wellbeing}, and \textit{Industry Sustainability}. These outcome variables are represented with blue boxes in the \cref{FCMdiagram}.


\begin{table}[H]
\centering
\resizebox{\textwidth}{!}{
\begin{threeparttable}
\caption{Network analysis indices \& results of the baseline quasi-dynamic analysis}
\label{degreeFCM}
\begin{tabular}{lcccc} \hline \hline
Variable & In-degree & Out-degree & Centrality & Steady state value\\ \hline
innovation                 &      4.44 &       2.02 &        6.45 &    0.97 \\
fieldExtraction            &      3.22 &       1.96 &        5.18 &    0.94 \\
palvalorisation            &      3.19 &       4.69 &        7.88 &    0.94 \\
funding                    &      1.79 &       1.27 &        3.07 &    0.85 \\
palProductsDemand          &      1.77 &       1.83 &        3.60 &    0.85 \\
pineappleProdProfitability &      2.39 &       0.98 &        3.37 &    0.82 \\
laborProductivity          &      0.66 &       0.57 &        1.23 &    0.81 \\
industrySustainability     &      1.75 &       0.60 &        2.35 &    0.81 \\
landAvailable              &      0.64 &       0.53 &        1.17 &    0.80 \\
academia                   &      0.67 &       0.60 &        1.27 &    0.80 \\
employment                 &      0.50 &       1.25 &        1.75 &    0.78 \\
industryImage              &      0.79 &       0.53 &        1.32 &    0.77 \\
businessRisk               &      0.50 &       0.54 &        1.04 &    0.77 \\
pineappleProdProductivity  &      0.53 &       0.62 &        1.14 &    0.77 \\
industryCustoms            &      0.50 &       0.60 &        1.10 &    0.76 \\
pollution                  &      0.75 &       0.74 &        1.49 &    0.70 \\
collabComms                &      1.28 &       0.77 &        2.04 &    0.66 \\
stubbleMgmtRegulation      &      0.69 &       2.98 &        3.68 &    0.66 \\
costFFmaterials            &      0.57 &       1.08 &        1.65 &    0.66 \\
ranchersProductivity       &      0.71 &       0.00 &        0.71 &    0.58 \\
communityHealth            &      2.15 &       0.79 &        2.93 &    0.57 \\
soilFertil                 &      0.37 &       0.00 &        0.37 &    0.55 \\
stableFly                  &      1.16 &       1.48 &        2.63 &    0.36 \\
agrochemicalsUse           &      3.03 &       0.75 &        3.78 &    0.20 \\
companySize                &      0.00 &       1.30 &        1.30 &    0.00 \\
govtPresence               &      0.00 &       1.36 &        1.36 &    0.00 \\
importRegulations          &      0.00 &       0.68 &        0.68 &    0.00 \\
industryTransparency       &      0.00 &       0.61 &        0.61 &    0.00 \\
rain                       &      0.00 &       0.61 &        0.61 &    0.00 \\
unevenTerrain              &      0.00 &       0.60 &        0.60 &    0.00 \\
intInstability             &      0.00 &       0.57 &        0.57 &    0.00 \\
greenConsumers             &      0.00 &       1.13 &        1.13 &    0.00 \\
 \hline \hline
\end{tabular}
\end{threeparttable}%
}
\end{table}

\subsection{FCM model and fuzzy inference}

The interpretation of FCM outputs from the quasi-dynamic analysis is done by comparing the steady state values of concepts after stabilisation. The steady state was achieved in the 10th iteration, as shown in \cref{outputFCM0}. This output reflects the current perception of stakeholders about the pineapple sector in Costa Rica in the context of circularity driven by PAL valorisation. As expected, the drivers of the model go to zero. We added a self-reinforcing relationship in the matrix, as recommended by \cite{diniz2015mapping}, but the steady state of the remaining variables did not change and the drivers all reached the same value, not providing additional information. Most variables' steady state corresponds to their centrality, i.e., those with high centrality also have a large steady state value. Nevertheless, some remarks are in order. \textit{Collaboration/Communication, Stubble Management Regulation, Community's Health, Stable Fly} and \textit{Agrochemicals Use} are all relatively low compared to other variables with similar centrality, and \textit{Labour Productivity}, on the contrary, presents a relatively high steady state value. It is also important to note that the transmitters with the largest values, apart from the obvious and central ones (innovation, field extraction, and PAL valorisation), are \textit{Funding}, \textit{Labour Productivity}, and \textit{Academia}. 


\begin{figure}[H]
\caption{The output of the quasi-dynamic analysis. The output for 11 concepts that go to zero is not shown.}  
\label{outputFCM0}
\centering
\includesvg[width=\textwidth]{fig/naturalSimulation.svg}
\end{figure}


\subsection{Drivers' intervention simulation}

The baseline output is useful to analyse what stakeholders believe is the unaltered result of the system's dynamic. This does not mean, for example, that the large values of \textit{Innovation} necessarily translate to a current situation of extensive innovation. Instead, it tells us that \textit{Innovation} is the central variable in the context of circularity and sustainability in the pineapple sector in CR. As such, we find it interesting and useful to analyse scenarios in which drivers are modified from the initial weights defined by stakeholders to see how the outcome variables react. We run the dynamic analysis under four different scenarios and compare the results in \cref{interTable}. Usually, scenarios are built by either modifying the initial values of the concepts (single-shot interventions) or by introducing a new concept in the initial FCM and defining the connection weight that it has on the target concepts (continuous interventions). We tested both types of scenario implementations and found no significant difference. The values shown are for the continuous interventions' implementation. Additional to the individual interventions, we run simulations with mixed interventions to see if there is a different effect when joining the stimuli of two drivers. The effect on the outcome variables for the individual interventions and the mixed interventions are depicted in \cref{interBars}.


\begin{figure}[ht]
\caption{Impact of interventions on outcome variables (\% Change from baseline's steady state)} 
\label{interTable}
\centering
\includesvg[width=\textwidth]{fig/interventionTable.svg}
\end{figure}


The intervention on \textit{Government presence} had the greatest effect on all four outcome variables by far, \textit{Industry Sustainability} receiving the largest impact, followed by \textit{Community's Health/Wellbeing}. The drivers \textit{Import regulations} and \textit{Green Consumers} also had a significant impact on \textit{Community's Health/Wellbeing}. The outcome variable that was less impacted by the single interventions was \textit{Profitability of the pineapple producers}. The intervention of \textit{Import regulations} only had a small effect on \textit{Community's Health/Wellbeing}, and the intervention of \textit{Industry's Transparency} had a negligible effect on outcome variables.  

Analysing \cref{interTable}, we notice that the effect of the single interventions on the transmitters was more heterogenous than on the outcome variables. The driver affecting more transmitters was \textit{Government presence}, followed by \textit{Green Consumers}. The transmitters affected the most were \textit{Agrochemical use}, \textit{Pollution}, and \textit{Collaboration/Communication}. The latter is only affected by \textit{Government presence} and \textit{Industry Transparency}, not by the intervention of \textit{Import Regulations} and \textit{Green consumers}. Interestingly, these two drivers reduce \textit{Pollution} significantly, which tells us that, even though stakeholders believe greater government involvement and industry transparency can improve collaboration, it would not reduce pollution significantly. Only regulations imposed by importing countries and the preferences of the consumers can change the behaviour of producers and, consequently, the level of pollution. These claims are supported by literature, although with some caveats. \cite{sajjad2015sustainable} explain that inadequate government support is one of the barriers to sustainable supply chain management implementation. As stated in a report by the \cite{oecd2017oecd}, this is the case in the Costa Rican agricultural sector, in which a fragmented 
institutional structure obstructs the coordination of actions and policy objectives. Moreover, the report acknowledges the  deficit in the technical capacity of the Agricultural Public Sector and its constraints in investment due to the intensification of budgetary restrictions since 2013. As regards consumer preferences, the benefits of going green ---increased efficiency in the use of resources, increased sales, development of new markets, improved corporate image, and enhanced competitive advantage--- are understood by companies \citep{dangelico2010mainstreaming}. As for the import regulations, evidence is mixed, but \cite{montiel2019effect} notes that the expansion of certifications creates uncertainties for producers that consequently reduce their readiness to adopt any standard.

\begin{figure}[h]
\caption{Impact of (mixed) interventions on outcome variables} \label{interBars}
\begin{subfigure}[b]{0.45\textwidth}
  \centering
  \includesvg[width=\textwidth]{fig/interventionBar.svg}
\caption{Impact of interventions on outcome variables} 
  \label{interventionBar}
\end{subfigure}%
  \hfill
\begin{subfigure}[b]{0.45\textwidth}
  \centering
  \includesvg[width=\textwidth]{fig/mixesOutput.svg}
\caption{Impact of mixed interventions on outcome variables}    
  \label{mixedBar}
\end{subfigure}
%\captionsetup{justification   = raggedright, singlelinecheck = false}
\end{figure}

We can also observe that the effect of \textit{Import regulations} and \textit{Green Consumers} on \textit{Community's Health/Wellbeing} is channelled via the reduction of \textit{Agrochemical Use}, while that of \textit{Government Presence} goes via two transmitters, \textit{Stubble management regulation} and \textit{Agrochemical Use}. We also find it interesting to note the negative impact on \textit{Business Risk} due to the intervention on \textit{Green consumers}. The direct connection between the concepts, which makes this impact large, is due to stakeholders indicating that \textit{Business Risk} is one of the main factors preventing investment in innovation related to PAL extraction and innovation. This is reasonable, as one of the main barriers when developing sustainability strategies is the uncertainty about the market demand \citep{chkanikova2015corporate}. Thus, we can see how more green consumption, which increases the demand for PAL products, alleviates this uncertainty and reduces business risk. 

As regards the mixed interventions, the first thing that strikes us by looking at \cref{mixedBar} is the greater effect that a combination of drivers can have on the outcome variables. The effect from the first three mixes can be mostly attributed to \textit{Government Presence}. The effect of the other three intervention mixes is almost negligible, except for the case of \textit{Community's Health/Wellbeing}. The combination of \textit{Government Presence} and \textit{Green Consumers} bring the greatest benefit to all outcome variables, highlighting how the influence of the market demand and the regulations can ensure collaboration and reduction of unsustainable practices at the same time. On the other hand, a relevant remark is that the percentage change from the steady state values attributed to these mixes is not much greater than the changes attributed to the single interventions. For example, the mix of \textit{Import Regulation} and \textit{Green Consumers} results in a change of 1.11\% from the steady state value of \textit{Community's Health}, whereas the changes from intervening these variables alone are 0.86\% and 0.75\% respectively. This is relevant, for instance, when deciding what policies or changes should be prioritised to attain sustainability in the industry, and for companies to know how external factors can affect their business. 


\subsection{Bringing it all together}

As we observe in the FCM diagram \cref{FCMdiagram} and by looking at the centrality in \cref{degreeFCM}, PAL valorisation is essential in the system.  Because of its centrality, it is challenging to understand how it can be enhanced to increase the impact on the outcome variables. The concept representing the valorisation of PAL was modelled neither as a driver nor as an outcome variable, precisely because it is the means to an end. If we were to run the quasi-dynamic analysis intervening \textit{PAL valorisation}, we would see a larger impact on the outcome variables than when intervening in the selected drivers (except for \textit{Government Presence}). But this is a fictitious intervention since PAL valorisation is not caused by itself, it needs to be ``activated" by a driver of change. Clearly, PAL valorisation can have a positive effect by substituting the use of agrochemicals, increasing employment, generating additional profit for producers, and improving the image of the industry, as has been modelled by the stakeholders. The quasi-dynamic analysis of the FCM provides a visual and numerical representation of this perception by stakeholders. Yet, stakeholders also represent the complexity of PAL valorisation in the FCM. Its impacts can be relevant and long-lasting, but they are channelled in a less direct manner than, for example, the import regulations. The prohibition of an agrochemical to export pineapple to a certain market has a simple, tangible effect, and its drivers and outcomes are trivial. But the motivators of change needed to valorise PAL, a valid alternative to agrochemical use, are less clear, and its consequences are more dispersed throughout the system. 

If we recall from the discussion on the definition of circular economy in \cref{theoryframe}, circular (bio)economy is a complex term drawing from different sustainability concepts to construct its strategies. For example, the use of agrochemicals in the linear economy serves a purpose: to maximise profits and minimise other resources such as water or labour. But the PAL valorisation, in its role as a CB solution, must meet more criteria, such as reducing waste (almost) completely and creating value-added products. If we  consider the features of CB in the agricultural sector, PAL valorisation practices should also account for the regeneration and biodiversity of the ecosystem that surrounds it. The complexity of these strategies and their consequences on the system are reflected in the perception of stakeholders on the system under study. 

At this point, we find it useful to try to answer the research questions defined in \cref{researchQ}. The initial question we proposed was
\textit{What are the cultural, financial, market-related, operational, and technological barriers preventing the valorisation of pineapple stubble in Costa Rica?}. These categories of barriers were tailored to the industry and country conditions, but they have similarities to those defined by \cite{gottinger2020studying} and summarised in \cref{theoryframe}. As these categories are commonly used when analysing the transition towards a circular bioeconomy, it delineating our discussion around them can help make comparisons in future research. 

First, we attempt to comment on the cultural aspects pertaining to the system described by the FCM. This falls into the category of \textit{Sectoral Routines and Structures}. As we can see in the \cref{FCMdiagram}, stakeholders view the customs of the industry as an impediment to valorising PAL. The customs of the industry are the inherited practices that prevail in the industry, such as the use of agrochemicals, monocropping, and productivity maximisation. These practices degrade the environment and do not align with sustainable practices \citep{magdoff2000hungry}, which is what stakeholders try to communicate. More interesting is to note the notion they have about how company size, profitability and customs are related. As explained by most stakeholders, the larger the company size, the larger the profits made, and the larger the profits, the more reinforced the customs. A systematic assessment of 118 studies explains that there is no conclusive evidence for a relationship between farm size and resource-use efficiency, GHG emissions, or profit \citep{ricciardi2021higher}. Nevertheless, it has been observed that small and medium-sized pineapple producers in Costa Rica have progressively been replaced by corporate farmers, who also benefit from the larger earnings \citep{rodriguez2020extractivismo}. Ultimately, the nature of the pineapple industry structure in Costa Rica---big corporations, monocropping, and productivity maximisation---incentivises less sustainable customs, which hinder the development of PAL valorisation as an alternative to current practices. 

Technologies required to extract PAL from the field and transform it into value-added products are not fully developed. Thus, funding for research and testing is still essential to make progress in circular-oriented innovation. Stakeholders identified company size as one of the factors increasing funding since large corporations are usually more capable of accessing international and domestic investment. We recognise that financial barriers to small and medium-scale farmers to valorise are present since they usually do not own machinery and cannot afford to invest in innovation. Large corporations that are predominant in the industry do have the means to invest in innovation. In fact, what seems striking from how stakeholders modelled the system is that there is no negative connection influencing funding. In this sense, we understand that the financial barrier is not related to access to funding per se, i.e., availability of funds, but to the mobilisation of investment resources. Although these considerations fall partially on the \textit{Market and Investment Conditions} category of barriers, the need to coordinate how much of the available funding is allocated to innovation and who should provide such funds shifts the financial barrier into a collaboration barrier. 

Each stakeholder---experts from the industry, the academia, and the government---has a different view about what roles each other should play in the PAL valorisation process, and who should initiate the required change. In the \cref{theoryframe}, we mentioned that circular strategies require linking knowledge and stakeholders in a manner not previously employed in the linear economy. The complexity of CE strategies and the dispersion of knowledge among stakeholders drives interdependence if there is alignment between motivations. Testing at scale the different PAL valorisation options is still uncommon. Collaboration in this respect is needed to share the potential risks and costs. Here, we recall the study by \cite{brown2019companies}, which tells us that there is a contradiction of companies wanting to share but also protect knowledge. Putting the business models into action can only happen if companies get out of the experimental phase. For this to occur, stakeholders need to acknowledge that the complexity of CB strategies requires collaboration in the technical and business model innovation. Risk-averse attitudes are a common obstacle to transitioning towards CB, as corroborated in the interviews. The stakeholders usually disagree on who should provide the required funds and take on the business risk. There is no consensus on who should be the initial investor, and what the role of the government should be to channel funds. What is evident is that most actors seem reluctant to take the first step without a clearer prospect of the results of the transition. The lack of collaboration and risk-averse attitude reinforce each other and create a combination of \textit{Knowledge and Network} and \textit{Sectoral Routines and Structures} barriers.

The technological barriers preventing PAL valorisation are closely related to operational barriers. As \cite{gottinger2020studying} indicate, the barriers category \textit{Technology and Materials} encompass difficulties to obtain input material, missing physical infrastructure, and technical barriers related to production and industrial application. From the FCM connections, the uneven terrain commonly present in pineapple fields is an operational barrier to extracting PAL from the field efficiently. Rainfall is another factor that, coupled with the uneven terrain, makes it very hard to manoeuvre any large machinery on the field. Different types of machinery have been tested, and others have been specifically made to extract PAL mechanically. Yet, to this day, there is no machine that can extract PAL without additional human labour in Costa Rica. As regards the valorisation options, several technologies to make value-added products exist. Yet, as stakeholders commonly mention, the difficulties to obtain input material to conduct large-scale projects is a barrier to the transition. Due to these operational barriers and the aversion to business risk mentioned above, valorisation research projects do not take place at a sufficiently large scale to extrapolate the results. This combination of technological and cultural barriers is easily understood by producers and people working closely with the logistical aspects of the business, but it can be less frequently identified by researchers and policymakers. 

The barriers identified by the stakeholders and modelled in the FCM give us an idea of what is preventing the valorisation of PAL from taking off. It is then natural to ask \textit{What is needed to overcome these barriers? Whose action is required?} In \cref{descriptiveFCM}, we mentioned that the connection between collaboration/communication and innovation has the lowest entropy, informing that actors agree strongly with its weight. Analysing how interventions to the drivers alter outcome variables, it is clear how government presence can positively influence collaboration and communication in the industry. This relation was identified mostly by pineapple producers, who see the government agencies as an intermediary capable of coordinating efforts from the industry, international organisations, and research institutions. It is interesting that, although some experts identified \textit{Policy and Regulation} barriers such as lack of technology push policies, most of them perceive the public agencies as potential drivers of change. Nevertheless, from the government's side, agencies of the Ministry of Agriculture are concerned with problems caused by the stable fly and stubble management practices. Thus, their coordination efforts are related to this matter, and not directly focused on PAL valorisation, which is seen as just another stubble management alternative. Moreover, although stakeholders picture the government as the necessary mediator, they acknowledge their lack of resources to monitor and lead a transformation in the industry structure. 

Another relevant concept affecting collaboration and communication in the FCM is the transparency/openness of the industry. This is one of the selected drivers to simulate an intervention in the system, but its effect on outcome variables is negligible. Yet, the simple idea that transparency about supply chains and openness from the companies can help to reduce environmental impacts is widely accepted \citep{kashmanian2017building, jahansoozi2006organization}. The reason this concept is closely connected to collaboration/communication in the context of PAL valorisation is the prevention of duplicated efforts to find solutions. As discussed in \cref{theoryframe}, there is a contradiction of companies wanting to share but also protect knowledge. If companies are open not only about their practices but also about their findings and innovations, collaboration is magnified to accelerate progress towards a common solution.

At a first glance, \textit{Research} can be an overlooked concept in the FCM diagram, as it is not connected to many other factors. As observed in the interviews, the subject of PAL valorisation has been predominantly conducted in academia, with few companies investing in experiments. As documented in \cref{neweconomy}, universities, usually from pineapple-producer countries, have done extensive research on the properties of the PAL to assess valorisation options. However, less research has focused on the operational and socioeconomic aspects of the process needed to valorise PAL at a large scale. Since access to machinery and land is limited to a few stakeholders, research on the engineering challenges to extracting PAL from the field is scarce. The producers who participated in the study mentioned the relevance of academia to bring about innovation and stressed the importance of collaborating with universities and other research institutions to undertake PAL valorisation projects. For researchers, as important as collaboration is access to funding. The only factor influencing \textit{Research} in the FCM is \textit{Funding}, which is a shared concern among researchers. In this sense, it is of the utmost importance to facilitate collaboration between researchers and companies with sufficient capital to undertake projects at a large scale. With the necessary funds and access to the industry's resources, more practical research can be conducted. This would knock down barriers in two groups, namely \textit{Market and Investment Conditions} and \textit{Technology and Material}. Additionally, knowledge from engineering companies and industries with supply chains similar to those of the pineapple can help to advance the development of pragmatic solutions to extract PAL from the field. Finally, partnerships with aid agencies and environmental organisations interested in participating in the development of sustainability programs can take over those tasks that government agencies cannot fulfil. If stakeholders collaborate and effectively distribute responsibilities among themselves, progress towards integration of PAL valorisation processes in the pineapple supply chain can start. 

After analysing the barriers to PAL valorisation and the actions needed to overcome them, one last question emerges: \textit{What are the benefits and challenges of valorising the stubble?}. Most benefits of PAL valorisation were mentioned by all stakeholders. We find it interesting that, as shown in \cref{entropyResults}, the entropy of \textit{PAL valorisation} to \textit{Use of agrochemicals}, to \textit{Stable fly}, and to \textit{Profitability of pineapple producers} is relatively high. Let us analyse these three connections. First, because agrochemicals are not only used for stubble management, but also to control pests, produce artificial ripening, and enhance fruit size, stakeholders view the potential of PAL valorisation to reduce agrochemicals as limited. Yet, any reduction in agrochemical use due to the valorisation of PAL is beneficial for the environment, as it reduces water, soil, and air pollution. More surprising is the large disagreement regarding the effect of PAL valorisation on the presence of the stable fly. The fly reproduces in the decomposing pineapple stubble after harvesting. If the PAL is extracted, and the remaining, low volume stubble is incorporated into the soil, the probability of stable fly reproduction is reduced significantly. Stable fly reduction is beneficial for several stakeholders in the system, including the pineapple producers, the ranchers, and the communities living close to the fields. Thus, a better understanding of the disagreement on its connection to PAL valorisation is needed. 

The third factor influenced by PAL valorisation with a large entropy is the profitability of pineapple producers. This connection occurs for three reasons: pineapple producers who take part in the PAL valorisation business can obtain additional profits; if the extraction of PAL benefits from economies of scale, its costs can eventually become lower than those of the current stubble management practices; freeing the fields from stubble allows pineapple producers to start the next plantation earlier, which translates into larger economic benefits. The latter consequence is also represented by the connection between \textit{PAL valorisation} and \textit{Land availability}. This connection is unquestionable, which is shown in its low entropy. Because this contradicts the large entropy of PAL valorisation to the profitability of pineapple producers and because of the clear benefits explained above, we venture to conclude that some stakeholders do not identify all economic benefits associated with PAL extraction and valorisation. In this sense, information campaigns on the benefits of PAL valorisation can increase the awareness of producers about this matter. The last benefit identified by stakeholders, the sustainability of the industry, has relatively low entropy. Despite the disagreements on the specific effects of PAL valorisation on different concepts, experts agree on the general idea that it can increase sustainability. 

When we think of the challenges to valorise PAL systematically, we immediately think of the title of this chapter. Harvesting The Fruits oparticipatory manner, and f Uncertainty refers to the complexity and fuzziness of the problem under study, but also to its challenges and possibilities. To clarify, by challenges we refer to milestones that need to be achieved and, in a way, they provide a more positive perspective than barriers. Overcoming, or adapting to uncertainty is perhaps one of the biggest challenges we identify in the study, which is translated into the concept \textit{Business risk} in the FCM. Stakeholders recognise the unpredictability of the market demand for bio-based products as one of the main challenges to investing in solutions. An increase in green consumerism can reduce uncertainty, as has been modelled in the FCM and represented in \cref{mixedBar}, but more market research in this area is needed to incentivise investment in PAL-valorisation solutions. Similarly, the introduction and changes of standards and regulations related to bio-based products, biofuels, and bioenergy which innovators must comply with make valorisation ventures less attractive. In this sense, we expect efforts to identify the market demand and the applicable regulations will provide clearer prospects to investors about the potential of PAL-based products. Another clear challenge is to attain a collaborative network of producers, researchers, government agencies, and investors that exchange knowledge and share risks and successes. We mentioned the barriers to and benefits of collaboration. The challenge is for every actor in the system should CE objectives be attained. Perhaps if researchers and producers manage to increase motivation and change the culture in the industry, societal transformations through larger-scale collaborations will take off. Finally, it is worth mentioning a more technical challenge, the consequences of stubble extraction on soil fertility which have not been quantified as of now. Stubble generates several environmental problems when managed in the field, but it also provides nutrients to the soil when incorporated into it. As PAL starts to be extracted, soil fertility can be reduced, affecting the productivity of the farmers. If it turns out that the effects of a systematic extraction are significant, pineapple producers will have to find solutions that do not require further use of agrochemicals. 

\section{Conclusion, Limitations, and Recommendations}

In this chapter, we have discussed the theories on the transition towards a circular (bio)economy (CB) and on Circular-Oriented Innovation (COI) relevant to the subject of Pineapple Leaves (PAL) valorisation. By means of the Fuzzy Cognitive Mapping (FCM) method, we structured and elicited the knowledge gathered through interviews with stakeholders pertaining to or related to the pineapple industry in Costa Rica. Our results show that advances in PAL valorisation are slow and have not developed significantly in the last two decades, since the pineapple industry started to grow rapidly. 

The study of the system helped us raise some interesting results. First, we observe stakeholders perceive that greater government involvement and industry transparency would improve collaboration, but it would not reduce pollution significantly. Only regulations imposed by importing countries and green consumerism can change the behaviour of producers and, consequently, the level of pollution. It is also relevant to remark on the little effect increase that mixed policy interventions have on the receiver concepts as compared to their single counterparts.

PAL valorisation, in its role as a CB solution, must meet more criteria than innovative solutions in the linear economy. Creating an added-value product while closing loops and protecting the biodiversity in the agricultural fields create a relevant challenge to make the transition towards a systematic PAL valorisation. The main cultural barriers that fall into the \textit{Sectoral Routines and Structures} category relate to the unsustainable practices that prevail in the industry, such as agrochemicals, monocropping, and productivity maximisation. As for the financial barriers, it was identified that funding for research and testing is still essential to make progress in circular-oriented innovation. Yet, more important seems to be the barriers related to collaboration, as there is no consensus on who should lead the experimentation phase and who should allocate funds to it. Lack of collaboration and risk-averse attitude reinforce each in the system.

The rain and the uneven, difficult terrain distinctive of the area were determined as operational and technological barriers. These barriers make the development of machinery capable of extracting large quantities of PAL a challenge. Consequently, input material cannot be obtained to conduct large-scale experiments. Coupled with risk aversion, these technological barriers prevent progress in innovation. 

Most stakeholders view the governmental agencies as potential drivers of change, instead of generators of barriers. They also agree on the needed increase in transparency and openness from the pineapple producers and innovators. This would help transfer knowledge, avoid duplication of efforts, and share risks. Companies show a contradiction in their desire to collaborate in this sense. Additionally, we find that research is seen as a relevant actor to drive circular-oriented innovation, but more research focused on the operational and socioeconomic aspects of the process needed to valorise PAL at a large scale is needed. Access to funding and input material is key in this respect.

The reduction of the stable fly as a benefit of valorising PAL is contested by some interviewees. More research into this disagreement would be useful to understand its causes. Another identified benefit is the reduction in agrochemicals, which is limited because of their use in other processes of pineapple production. Finally, although the transition towards a CB in the pineapple sector would bring larger profits to investors, this is not acknowledged by all stakeholders. Thus, we argue that it is needed to raise awareness about these benefits to motivate potential investors. 

Our results emphasise the early stage of development at which PAL valorisation in Costa Rica. Yet, the theories on COI help us illustrate that the barriers faced by the industry are common in CB transitions. As recommended by \cite{blomsma2022making}, we conclude it would be useful to take a sufficiently long-time horizon to understand circular phenomena in the PAL valorisation industry in Costa Rica. Periodic analyses by researchers to compare the evolution of the transition can shed light on features that are not identified in a one-time study. As for stakeholders and policymakers, we recommend looking at challenges that can be completed instead of at barriers that may not be overcome. Market research and a better understanding of the regulations pertaining to PAL-based solutions would provide clearer prospects for opportunities. Also, every actor in the system has the responsibility to connect with each other and create tighter and stronger collaborative networks. The transition towards a CB in the agricultural sector is not a one-day or one-person effort, but a collection of milestones achieved by collaboration throughout time. 

Finally, we find it relevant to mention several limitations and considerations of the study. The use of interviews is useful to understand the subject under study in depth, but limits the extent to which results from the analysis can be extrapolated. In some cases, the results are corroborated by the literature and the theory, but sometimes they are distinctive of the area under study. Regarding the use of FCM to elicit knowledge, although the stakeholders create the connections, the researcher is the one who ultimately decides how the map is constructed. In this sense, it is important to pay attention to potential biases and errors that may arise in the modelling process. For example, missing connections that are true by construction but that were not mentioned in the interviews were later identified, such as the negative influence of \textit{Agrochemical use} on \textit{Stable fly}. In this sense, we recommend researchers complement the interviews with a literature review to add essential connections to the system, as in \cite{edwards2021building}. Aside from these considerations, we find the FCM method useful to organise and identify fuzzy knowledge, especially in the case of unexplored or developing phenomena.

The collection of information in two stages, using one-to-one interviews and an online questionnaire proved relatively inefficient, with a low response rate in the latter. We believe this is related to stakeholders in the agricultural sector being used to working outdoors and with dynamic routines. Moreover, aggregating contrary views and simplifying a complex system modelled qualitatively can be challenging for the researcher. In this sense, we recommend using in-person workshops to motivate participation and reach agreements about complex connections. Finally, a clear bias in the information collection is the selection of interviewees, as these were selected because of their relation to PAL valorisation activities. If we were to interview pineapple producers who are not aware of this CB solution, we would perhaps gather less information, but also very useful and valid viewpoints. Finally, as more is understood about the structure of the PAL valorisation industry in Costa Rica, the use of a theoretical framework focused on circular bioeconomy transitions, such as the TIS, can prove useful to carry out periodic studies that can track progress throughout time in an organised manner. 
