\chapter{The Pineapple Leaves Route}
\label{FLPchapt4}

As discussed in \cref{chapter_3lab}, even though small-scale projects of Pineapple Leaves (PAL) valorisation have been conducted, there is no research on how a valorisation process at the industry scale would be carried out operationally. Minimising costs in the valorisation process is paramount should PAL-based products compete with conventional, fossil-fuel-based products. The process of valorising PAL can be divided into three main stages: extraction, transportation, and material transformation. In this chapter, we focus on the last two by developing a Facility Location Problem in which we try to locate and minimise the optimal number of material transformation plants conditional on the location of pineapple fields in Costa Rica, the costs of transporting PAL, and the costs of opening and running a hypothetical PAL valorisation facility. 

\section{Facility Location Problem}

The Facility Location Problem (FLP) is an optimisation problem that determines the best location for facilities to be placed based on facility costs, geographical demands, and transportation distances. Results drawn from an FLP are critical in strategic planning for private and public entities. Because of the high costs of property acquisition and facility construction, facility location decisions are long-term, strategic investments. The FLP became relevant due to the industrial revolution, as the development of rail transport, energy, and urban growth offered more options for distributing firms and their operations. Alfred Weber first developed a theory of location problems with his publication \textit{Über den Standort der Industrie} (Theory of the Location of Industries) in 1909, in which he modelled an optimal location and minimal cost for manufacturing plants taking into account several spatial factors \citep{fearon2006alfred}. Since the sixties, when \citeauthor{hakimi1964optimum} published his work on an FLP for switching centres in a communications network and police stations in a highway system, there have been numerous studies on different types of FLP \citep{farahani2009facility}.

Location problems consist of four main components: existing demand points, some facilities which are supposed to serve the demand points, a feasible solution space in which the demand points and facilities are dispersed, and a measurement criterion that explains distances (e.g., time or cost) between facilities. Although there are many versions of the FLP applied to different types of problems, they are all comprised of an objective function and a set of constraints. The many ways in which FLP can be categorised are beyond the scope of our study, but \cite{wolf2022solving} provide a list of the most important categorisations. First, we can consider private and public sector location problems. In the former, the objective functions are profit maximisation or cost minimisation, while the public sector problems also consider nonmonetary costs and benefits, such as environmental costs when locating hazardous waste repositories or the value of saved lives when establishing emergency centres. 

The second classification is planar versus network location problems: In planar FLP, the locations of a finite number of demand points as well as of the optimal facilities may be sparsed everywhere in the Euclidean plane. On the contrary, network location problems are defined on networks composed of nodes and edges. Almost all network location problems assume that the demand and facility points coincide with the vertices of a network and that transport occurs only along the edges of this network. The weights assigned to the edges can specify not only distances, but also travel times, or transportation cost. These terms might remind us of chapter \cref{chapter_3lab}, in which we modelled a Fuzzy Cognitive Map (FCM) composed of concepts (nodes) that affect each other through the weighted edges of a network. In the FLP, we study a different problem, but it is still interesting to highlight how graph theory is used in a wide range of applications \citep{papageorgiou2003fuzzy, seppanen1970facilities}. Both planar and network FLP can be continuous, meaning that the generation of feasible sites are left to the model at hand, or discrete space problems, in which facility candidates are selected a priori. Even when problems are continuous by nature, most of the results in the literature are discretised.

FLP can also be categorised into capacitated or uncapacitated problems. Capacitated facilities have a constrained capacity to serve the demand sites, while uncapacitated facilities are unrestricted. A fourth relevant classification of FLP is when we consider solving for desirable or undesirable facilities. Most problems locate desirable facilities, such as warehouses, service centres, or hospitals as close as possible to the demand points. On the contrary, when dealing with undesirable facilities, such as landfills, polluting plants, etc., the objective function of the FLP is to maximise the weighted distance function between the facilities and the served demand points. Finally, we consider the classification of FLP by the number of facilities to be located. When the number of facilities is specified exogenously, the problem can be either single or multi-facility. On the contrary, FLP can also be defined with an output parameter of the number of facilities to be optimised. It is important to note that the quantity of facilities influences the execution time of any algorithm. In complexity theory, the general problem of locating optimal facilities in a network is NP-hard. NP, which stands for nondeterministic polynomial, is a set of problems whose solutions can be verified in polynomial time. Yet, it is unknown whether NP-hard problems have an algorithm for finding the solution in polynomial time; this question is known as the P versus NP problem. NP-hard problems are at least as hard as the hardest problems in NP and are considered to be some of the most difficult problems to solve using algorithms \cite{kokash2005introduction, cooper1963location}.

In operations research, as well as in other fields, optimisation problems are defined as NP-hard problems. Only when the problem at hand is small enough (e.g., \cite{sridharan1995capacitated} indicate 50 facilities and 50 demand points) it can be handled by using exact mathematical methods. Thus, researchers came up with heuristic and metaheuristic methods, which are approximation methods that can find a good enough solution in a reasonable time. Heuristic methods are usually defined for the particular problem it seeks to solve, and can become insufficient for other problems. Metaheuristic methods, on the contrary, are generic, problem-independent algorithms that can be adapted to almost all optimisation problems. Exact methods find the optimal solution, but they are computationally intensive and impractical for large problems. (Meta)heuristic methods, on the other hand, overcome the NP-hardness of the optimisation problems by finding a good enough solution quickly and efficiently, but may not guarantee an optimal solution. The most common metaheuristic methods are simulated annealing, tabu search, genetic algorithm, variable neighbourhood search, and ant systems. All of these are designed to decrease the probability  of falling in local optimal \citep{abdel2018metaheuristic}.

The use of FLP for waste management is common in the literature. With the adoption of a linear economy and a throwaway culture, waste generation increased and its disposal became a relevant problem all around the world. Operations research provides the tools needed to optimise waste disposal and minimise the costs and environmental degradation caused by waste management. \cite{adeleke2020facility} provide a good summary of existing FLP models and optimization techniques and their application to solid waste management problems. Most studies focus on the treatment of municipal, industrial, healthcare, and hazardous waste. As waste can have more than one disposal site, it is important to consider other facilities associated with the collection sites, such as recycling centres and landfills. This also applies when implementing valorisation processes to the residues. Focusing on the uncertainty and robustness of the optimisation, \cite{berglund2014robust} implement both exact and metaheuristic methods to solve an FLP of hazardous waste. \cite{hu2017bi} developed a facility location model to locate waste-to-energy facilities and that minimises government spending and environmental adverse effects. Another good example of an FLP that takes environmental costs into account is the study on waste collection in China by \cite{wu2020optimization}, which considers greenhouse gas emission costs and conventional waste management costs.  Since in this study we deal with a FLP for residues valorisation, it is relevant to mention the previous research focused in this area. \cite{athira2020effective} used a location-allocation analysis to optimise the location of new cement plants based on availability of sugarcane bagasse ash produced in the sugar industry, which can serve as a supplementary and partial alternative to cementitious material. \cite{guerrero2016gis} assessed the use of banana crops residues in the generation of bioethanol and identified two optimal locations for energy conversion facilties in Ecuador, a big banana producer country. Another example of biorefineries location optimisation is the study by \cite{duarte2014facility}, who applied a mixed-integer linear programming formulation to locate a second-generation bioethanol plant fed with coffee cut stems in Colombia. 


\citep{}

\section{Description of the study area}

\section{Methodology}

\subsection{Source of PAL}

1. We take the north and east of the country for the case study, specifically Huetar Norte and Huetar Caribe, which together produce 86\% of the pineapple in the country. 

2. I specify the calculations of the stubble supply: The source of stubble was determined using the MOCUPP database on pineapple production in the country. The polygons of the fields were taken to calculate the area and make an approximation of the stubble supply coming from each location. 

3. For the calculation, the centroid of the polygon was snapped to the closest road network edge to ease the analysis. To account for transport within the field, a penalty was applied according to the polygon area. 

4. Show a map of the area with the fields (polygons) selected for the study. 

5. We are assuming that this is a collective solution, in the sense that we are optimizing for the stubble supply regardless of ownership. We can theorize that this would be possible in a scenario of an external company getting/buying the stubble from the farmers, or of a solution managed by a cooperative. 


\subsection{Defining the potential locations for processing plants}

1. What type of processing facility are we considering, since there is no existent valorisation process we can use as an example? The decision is left to my criteria, and I would decide on a single- or mixed-process facility and argue why this is a sensible option. Biogas or bioethanol are good options, and these could be partially combined with extraction for fibre. 

3. The potential locations of facilities can be determined in many ways. In the literature, this selection is many times not explained. In other cases, the reasoning seems arbitrary, or simply tailored for the purpose of the study. A clever way of defining facilities for management of residues is using the applicable zoning regulation. This works for relatively well, as you can simply select the areas in which it is allowed to build industrial facilities.

3. I would give some details about how the zoning regulation works in CR. It works by cantón and some have a plan, others are in the process of developing it. 

4. Although using zoning areas is ideal to limit the potential locations, in the case of CR this is not possible, as there are no zoning maps for the country (This would be very useful for planning of any type in the country). Thus, I decided to use a grid to discretize the continuous space and locate clusters along the rode network in which facilities can be located. I explain the details and consequences of applying this method.

5. I show a map of how the potential facilities are distributed.


A candidate facility should be a location that is suitable for the event or structure you are locating. For instance, if you are locating distribution centers, first, you might need to find parcels that are for sale, within your budget, properly zoned, and large enough to contain the distribution center you plan to build. You might also choose to include parcels that already have structures on them that are large enough to house your distribution center. There is no limit to the number of factors you might consider in determining suitability for your facilities.

(https://desktop.arcgis.com/en/arcmap/latest/extensions/network-analyst/location-allocation.htm)


\subsection{Optimisation Algorithm}

1. Here I explain what type of meta-heuristic model I used (simulated annealing or tabu search), and explain its properties.

2. I explain what the objective function and the constraints are:

3. We want to find the location and number of facilities to allocate the supply of stubble at the lowest cost and CO2 emissions. Inevitably, the distance will be minimised accordingly. There are two sources of costs and emissions: transport and facilities (construction and running). 

4. On the other hand, we can account for costs and emissions reduction from not managing the stubble in the field, although these numbers would be rough estimates. 

Total costs = Cost of processing plants + transport costs - stubble management costs 

Emissions =  Emissions of processing plants + transport emissions - stubble management emissions

\paragraph{Emissions and costs reduction} \

1. If I decide to include emissions and costs reductions: 

2. Costs reductions = cost of agrochemicals and machinery used for decomposition

3. Emissions reduction = CO2 emissions from machinery fuel and agrochemicals 

\paragraph{Transportation costs and emissions} \

1. Emissions from transport = emission of transport of one tonne per km X distance from the stubble location to the facility X stubble quantity at source

2. Cost of transport = cost of transport of one tonne per km X distance from the stubble location to the facility x stubble quantity at source

\paragraph{Processing costs and emissions} \

1. These are dependent on the total amount of stubble processed in the facility. We need a function to represent the economies of scale.

2. Estimates of biogas/bioethanol plants from the literature are used. (the ICE doesn't have open-access data from the biogas plant they built, and the person I contacted who is building one with Nicoverde didn't reply).

3. Here I need to account for fixed and variable costs. I am still confused on how to determine this: I can probably use the life-span of a plant to make all calculations.

4. Processing emissions can be drawn in the same fashion. If only energy consumption is available, an emission factor can be used to calculate the corresponding emissions.

%\paragraph{Optimisation parameters}

%1. A meta-heuristic model can run forever unless certain parameters are used to limit the search. 

\section{Repeatability and sensitivity analysis} 

1. Since the algorithm from the meta-heuristic is stochastic, different values can arise using the same model. Thus, we need to repeat the calculation several times.

2.We change the parameters values of the objective function decreasing and increasing by a percentage (e.g. +- 15\%)

3. Firstly, assessing the sensitivity of the model allows us to estimate how uncertainties in parameters change the rest of the model. Also, by assessing the sensitivity of the model to changes in certain parameters, we can evaluate how interventions in the system can affect the final distribution of facilities.

\section{Results}

1. Present the results of the optimisation problem, including the optimal solutions and their properties.

2. Several plots would be provided to accompany the results and the sentivity analysis.

\section{Discussion}

1. Discuss the implications of the findings and their relationship to the research question

2. Discuss what the results imply for the case of PAL in CR.

3. Evaluation of the advantages and limitations of the FLP for the case of PAL in Costa Rica

4. Potential directions for future research 

\section{Conclusion and Recommendations}

1. I give a conclusion of the method used and how good it fit my case study.

2. I comment on the main findings of the FLP and how it helped reduce uncertainty about where to start with the valorisation cost-effectively.

3. I reiterate how the limitations can be tackled in future research.

\begin{comment}


\section{Supply Chain Management Optimisation}

The use of agricultural waste in valorisation processes requires investment in processing plants and costs in transportation. It also generates GHG emissions determined by the valorisation option process (e.g. biogas plant or fibre extraction), the distance and mode of transportation, the extraction, the storage, and the manufacturing of machinery. On the other hand, as the current management of pineapple stubble requires the use of agrochemicals, its valorisation can also reduce GHG emissions as it would eliminate the need to manage the stubble in the field. 

As the PAL is a residue produced by the the pineapple production, its production does not compete with other crops or land uses. Thus, the optimisation of its supply chain is purely based on the minimisation of costs and GHG emissions. The total cost of a potential PAL supply chain is composed of capital costs of the valorisation plant, operation costs of the valorisation process, and PAL costs. The latter is composed of extraction, transportation, and storage.

The PAL costs depend on the spatial conditions such as the availability of 
PAL and the road network for transportation.  The optimisation of the PAL supply chain can be seen as the analysis of the spatial distribution and amount of potential PAL for different valorisation options and the optimal locations, sizes and number of processing plants. Thus, a GIS-based approach is proposed  as it  can assist in the location selection process. 


\section{Primary and Secondary Data Collection}

Data collection from academic and grey literature, interviews, and observations will be conducted to document   the PAL valorisation options being considered in CR. It is expected to identify the (potential) demand and price of PAL by-products and determine the regulations that may affect the valorisation options.

\end{comment}